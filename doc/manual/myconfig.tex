% $HeadURL$
% $Id$

% ----------------------------------------------------------
% 
% ----------------------------------------------------------

% A sane version of verb/texttt (requires url package)
\newcommand{\mytt}{\begingroup \urlstyle{tt}\Url}
\newcommand{\mysf}{\begingroup \urlstyle{sf}\Url}

\newcommand{\myfiguretextsize}{small}

\newcommand{\NIL}{\fnnm{NIL}}

% Because the retarded tabbing environment limits you to about 12 tabs
%\newcommand{\TB}{\hspace{.2in}}
\newcommand{\TB}{\hbox to 3ex{\hfill}}

% ----------------------------------------------------------
% my common text commands
% ----------------------------------------------------------

\newcommand{\ie}{i.e.}         % followed by a comma
\newcommand{\eg}{e.g.}         % followed by a comma
\newcommand{\cf}{cf.\@}        % always followed by a space
\newcommand{\Cf}{Cf.\@}        % always followed by a space
\newcommand{\vs}{\emph{vs.}\@} % always followed by a space
\newcommand{\etc}{etc.}        % 
\newcommand{\etal}{et al.\@}
\newcommand{\viz}{\emph{viz.}}
\newcommand{\perse}{per se}
\newcommand{\naive}{na\"{i}ve}
\newcommand{\nee}{n\'{e}e}

\newcommand{\textseparator}{%
  \begin{center}{\textasteriskcentered \hspace{2em} \textasteriskcentered \hspace{2em} \textasteriskcentered}\end{center}} % \textbullet


% ----------------------------------------------------------
% tools
% ----------------------------------------------------------

\newcommand{\HPCToolkit}{\textsc{HPCToolkit}}
\newcommand{\hpctoolkitorg}{\href{http://hpctoolkit.org}{\textsc{HPCToolkit}}}
\newcommand{\hpcrun}{\texttt{hpcrun}}
\newcommand{\hpclink}{\texttt{hpclink}}
\newcommand{\hpcstruct}{\texttt{hpcstruct}}
\newcommand{\hpcprof}{\texttt{hpcprof}}
\newcommand{\hpcprofmpi}{\texttt{hpcprof-mpi}}
\newcommand{\hpcprofAll}{\texttt{hpcprof/mpi}}
\newcommand{\hpcviewer}{\texttt{hpcviewer}}

% ----------------------------------------------------------
% 
% ----------------------------------------------------------

% use textsf instead of mathsf
\newcommand{\fnnm}[1]{\ensuremath{\textsf{#1}}}


% ----------------------------------------------------------
% theory of sampling-based measurement
% ----------------------------------------------------------

\newcommand{\population}[1]{\ensuremath{\mathcal{P}_{#1}}} % mathcal, mathfrak

\newcommand{\populationmap}{\ensuremath{\mathcal{M}}}
\newcommand{\contextmap}{\ensuremath{\mathcal{C}}}
\newcommand{\instructionproj}{\ensuremath{\mathcal{I}}}

\newcommand{\resourceReqSat}{\ensuremath{r_{\fnnm{sat}}}}
\newcommand{\resourceReqMax}{\ensuremath{r_{\fnnm{max}}}}


% ----------------------------------------------------------
% measurement: hpcrun binary analysis algorithm
% ----------------------------------------------------------

\newcommand{\CanonFrmSym}{\ensuremath{\mathbf{C}}}


% ----------------------------------------------------------
% attribution: binary analysis algorithm
% ----------------------------------------------------------

\newcommand{\NonOverlapRef}{Non-overlapping Principle}
\newtheorem*{thmnonoverlap}{Non-overlapping Principle}
\newcommand{\nested}{\ensuremath{\sqsubset\!\!\!\prec}}
%\newcommand{\nested}{\ensuremath{\ll}}
%\newcommand{\nested}{\ensuremath{\Subset}}
%\newcommand{\nested}{\ensuremath{\langle<}}

\newcommand{\ProcThmRef}{Procedure Invariant}
\newcommand{\LoopThmRef}{Loop Invariant}
\newtheorem{thmproc}{\ProcThmRef{}}
\newtheorem{thmloop}{\LoopThmRef{}}

\newcommand{\minmax}{\textsf{min-max}}
\newcommand{\bbranchmax}{\textsf{bbranch-max}}
\newcommand{\bbranchmaxfuzzy}{\textsf{bbranch-max-fuzzy}}
\newcommand{\bbranchmaxfuzzySAVE}{\textsf{bbranch-max-fuzzy}}


\newcommand{\instantiateDWARFdescriptors}{\textsf{instantiate-DWARF-descriptors}}
\newcommand{\coalesceduplicatestatements}{\textsf{coalesce-duplicate-statements}}
\newcommand{\mergeperfectlynestedloops}{\textsf{merge-perfectly-nested-loops}}
\newcommand{\partitionaliencontexts}{\textsf{partition-alien-contexts}}
\newcommand{\removeemptyscopes}{\textsf{remove-empty-scopes}}


% ----------------------------------------------------------
% logical call path profiling
% ----------------------------------------------------------

\newcommand{\note}{note}
\newcommand{\notes}{notes}
\newcommand{\pnote}{\emph{p}-note}
\newcommand{\lnote}{\emph{l}-note}
\newcommand{\pnotes}{\emph{p}-notes}
\newcommand{\lnotes}{\emph{l}-notes}

%\newcommand{\pnotesym}[2]{\ensuremath{p_{#1,#2}}}

\newcommand{\bichord}{bichord}
\newcommand{\bichords}{bichords}
\newcommand{\bichordsym}[1]{\ensuremath{\langle P_{#1}, L_{#1} \rangle}}
\newcommand{\bichordnotessym}[3]{\ensuremath{\langle \pchordsym{#1}{#2}, \lchordsym{#1}{#3} \rangle}}
\newcommand{\pchordsym}[2]{\ensuremath{(p_{#1,1}, \ldots, p_{#1,#2})}}
\newcommand{\lchordsym}[2]{\ensuremath{(l_{#1,1}, \ldots, l_{#1,#2})}}

\newcommand{\pchord}{\emph{p}-chord}
\newcommand{\lchord}{\emph{l}-chord}
\newcommand{\pchords}{\emph{p}-chords}
\newcommand{\lchords}{\emph{l}-chords}

\newcommand{\stepbichord}{\textsf{step-bichord}}
\newcommand{\steppchord}{\textsf{step-pchord}}
\newcommand{\steppnote}{\textsf{step-pnote}}
\newcommand{\steplnote}{\textsf{step-lnote}}

\newcommand{\assocM}{\ensuremath{\mathbf{M}}}
\newcommand{\assocOneToOne}{\ensuremath{1 \leftrightarrow 1}}
\newcommand{\assocOneToZero}{\ensuremath{1 \leftrightarrow 0}}
\newcommand{\assocMToZero}{\ensuremath{\assocM{} \leftrightarrow 0}}
\newcommand{\assocMToOne}{\ensuremath{\assocM{} \leftrightarrow 1}}
\newcommand{\assocOneToM}{\ensuremath{1 \leftrightarrow \assocM{}}}

\newcommand{\assocclassAToZero}{\ensuremath{\mathcal{A} \leftrightarrow 0}}
\newcommand{\assocclassAToOne}{\ensuremath{\mathcal{A} \leftrightarrow 1}}
\newcommand{\assocclassOneToA}{\ensuremath{1 \leftrightarrow \mathcal{A}}}

\newcommand{\cctNodeStruct}{\texttt{Node}}
\newcommand{\cctNodeStructFull}{\cctNodeStruct{}-structure}

\newcommand{\fnAssoc}{\fnnm{assoc}}
\newcommand{\fnAssocEq}{\fnnm{assoc}\ensuremath{\mathsf{=}}}
\newcommand{\fnAssocNeq}{\fnnm{assoc}\ensuremath{\mathsf{\neq}}}

\newcommand{\fnAssocClass}{\fnnm{assoc-class}}
\newcommand{\fnAssocClassEq}{\fnnm{assoc-class}\ensuremath{\mathsf{=}}}
\newcommand{\fnAssocClassNeq}{\fnnm{assoc-class}\ensuremath{\mathsf{\neq}}}

\newcommand{\fnIp}{\fnnm{ip}}
\newcommand{\fnIpEq}{\fnnm{ip}\ensuremath{\mathsf{=}}}

\newcommand{\fnLip}{\fnnm{lip}}
\newcommand{\fnLipEq}{\fnnm{lip}\ensuremath{\mathsf{=}}}

\newcommand{\fnNoteId}{\fnnm{note-id}}
\newcommand{\fnNoteIdEq}{\fnnm{note-id}\ensuremath{\mathsf{=}}}

\newcommand{\fnSharable}{\fnnm{sharable?}}

\newcommand{\CilkSpawn}{\texttt{spawn}}
\newcommand{\CilkSync}{\texttt{sync}}

% ----------------------------------------------------------
% scalable analysis and presentation
% ----------------------------------------------------------

\newcommand{\metricFnInit}{\ensuremath{\text{\textbigcircle}}}  % \varbigcirc [bbding]
\newcommand{\metricFnAccum}{\ensuremath{\bigodot}}
\newcommand{\metricFnCombine}{\ensuremath{\bigoplus}}
\newcommand{\metricFnFini}{\ensuremath{\text{\raisebox{-0.35ex}{\Large\CIRCLE}}}} % \textbullet


%\newcommand{\metricFormula}{\text{{\it\Large f}{\scriptsize\ }}}
\DeclareMathOperator*{\metricFormula}{\text{{\it\Large f}}}

\newcommand{\cctProcFrm}{\textsf{ProcFrame}}
\newcommand{\cctLoop}{\textsf{Loop}}
\newcommand{\cctAlien}{\textsf{Alien}}

\newcommand{\cctCallSite}{\textsf{CallSite}}
\newcommand{\cctStmt}{\textsf{Stmt}}

\newcommand{\fnDescStmt}{\fnnm{desc-}\cctStmt{}}
\newcommand{\fnChildStmt}{\fnnm{child-}\cctStmt{}}
