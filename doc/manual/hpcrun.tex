% -*- Mode: latex; -*-
% $HeadURL$
% $Id$

This chapter describes the mechanics of using \hpcrun{} and \hpclink{}
to profile an application and collect performance data.  For advice on
choosing sampling events, scaling studies, etc, see
Chapter~\ref{chpt:effective-performance-analysis} on {\it Effective
Strategies for Analyzing Program Performance}.

\section{Using \hpcrun{}}

The \hpcrun{} launch script is used to run an application and collect
performance data for {\it dynamically linked\/} binaries.  For
dynamically linked programs, this requires no change to the program
source and no change to the build procedure.  You should build your
application natively at full optimization.  \hpcrun{} inserts its
profiling code into the application at runtime via \verb|LD_PRELOAD|.

The basic options for \hpcrun{} are \verb|-e| (or \verb|--event|) to
specify a sampling source and rate and \verb|-t| (or \verb|--trace|) to
turn on tracing.  Sample sources are specified as `\verb|event@period|'
where \verb|event| is the name of the source and \verb|period| is the
period (threshold) for that event, and this option may be used
multiple times.  Note that a higher period implies a lower rate of
sampling.  The basic syntax for profiling an application with
\hpcrun{} is:

\begin{quote}
\begin{verbatim}
hpcrun -t -e event@period ... app arg ...
\end{verbatim}
\end{quote}

For example, to profile an application and sample every 15,000,000
total cycles and every 400,000 L2 cache misses you would use:

\begin{quote}
\begin{verbatim}
hpcrun -e PAPI_TOT_CYC@15000000 -e PAPI_L2_TCM@400000 app arg ...
\end{verbatim}
\end{quote}

The units for the \verb|WALLCLOCK| sample source are in microseconds,
so to sample an application with tracing every 5,000 microseconds
(200~times/second), you would use:

\begin{quote}
\begin{verbatim}
hpcrun -t -e WALLCLOCK@5000 app arg ...
\end{verbatim}
\end{quote}

\hpcrun{} stores its raw performance data in a {\it measurements}
directory with the program name in the directory name.  On systems
with a batch job scheduler (eg, PBS) the name of the job is appended
to the directory name.

\begin{quote}
\begin{verbatim}
hpctoolkit-app-measurements[-jobid]
\end{verbatim}
\end{quote}

It is best to use a different measurements directory for each run.
So, if you're using \hpcrun{} on a local workstation without a job
launcher, you can use the `\verb|-o dirname|' option to specify an
alternate directory name.

For programs that use their own launch script (eg, \verb|mpirun| or
\verb|mpiexec| for MPI), put the application's run script on the
outside (first) and \hpcrun{} on the inside (second) on the command
line.  For example,

\begin{quote}
\begin{verbatim}
mpirun -n 4 hpcrun -e PAPI_TOT_CYC@15000000 mpiapp arg ...
\end{verbatim}
\end{quote}

Note that \hpcrun{} is intended for profiling dynamically linked {\it
binaries}.  It will not work well if used to launch a shell script.
At best, you would be profiling the shell interpreter, not the script
commands, and sometimes this will fail outright.

It is possible to use \hpcrun{} to launch a statically linked binary,
but there are two problems with this.  First, it is still necessary to
build the binary with \hpclink{}.  Second, static binaries are
commonly used on parallel clusters that require running the binary
directly and do not accept a launch script.  However, if your system
allows it, and if the binary was produced with \hpclink, then
\hpcrun{} will set the correct environment variables for profiling
statically or dynamically linked binaries.  All that \hpcrun{} really
does is set some environment variables (including \verb|LD_PRELOAD|)
and \verb|exec| the binary.

% ===========================================================================

\section{Using \hpclink{}}

For now, see Chapter~\ref{chpt:statically-linked-apps} on {\it
Monitoring Statically Linked Applications}.

% ===========================================================================

\section{Sample Sources}

This section covers the details of some individual sample sources.  To
see a list of the available sample sources and events that \hpcrun{}
supports, use `\verb|hpcrun -L|' (dynamic) or set
`\verb|HPCRUN_EVENT_LIST=LIST|' (static).  Note that on systems with
separate compute nodes, you will likely need to run this on one of the
compute nodes.

\subsection{PAPI}

PAPI, the Performance API, is a library for providing access to the
hardware performance counters.  This is an attempt to provide a
consistent, high-level interface to the low-level performance
counters.  PAPI is available from the University of Tennessee at:

\begin{quote}
\url{http://icl.cs.utk.edu/papi/}
\end{quote}

PAPI focuses mostly on in-core CPU events: cycles, cache misses,
floating point operations, mispredicted branches, etc.  For example,
the following command samples total cycles and L2 cache misses.

\begin{quote}
\begin{verbatim}
hpcrun -e PAPI_TOT_CYC@15000000 -e PAPI_L2_TCM@400000 app arg ...
\end{verbatim}
\end{quote}

The precise set of PAPI preset and native events is highly system
dependent.  Commonly, there are events for machine cycles, cache
misses, floating point operations and other more system specific
events.  However, there are restrictions both on how many events can
be sampled at one time and on what events may be sampled together and
both restrictions are system dependent.  Table~\ref{tab:papi-events}
contains a list of commonly available PAPI events.

To see what PAPI events are available on your system, use the
\verb|papi_avail| command from the \verb|bin| directory in your PAPI
installation.  The event must be both available and not derived to be
usable for sampling.  The command \verb|papi_native_avail| displays
the machine's native events.  Note that on systems with separate
compute nodes, you normally need to run \verb|papi_avail| on one of
the compute nodes.

\begin{table}
\begin{center}
\begin{tabular}{|l|l|}
\hline
\verb|PAPI_BR_INS| & Branch instructions \\
\hline
\verb|PAPI_BR_MSP| & Conditional branch instructions mispredicted \\
\hline
\verb|PAPI_FP_INS| & Floating point instructions \\
\hline
\verb|PAPI_FP_OPS| & Floating point operations \\
\hline
\verb|PAPI_L1_DCA| & Level 1 data cache accesses \\
\hline
\verb|PAPI_L1_DCM| & Level 1 data cache misses \\
\hline
\verb|PAPI_L1_ICH| & Level 1 instruction cache hits \\
\hline
\verb|PAPI_L1_ICM| & Level 1 instruction cache misses \\
\hline
\verb|PAPI_L2_DCA| & Level 2 data cache accesses \\
\hline
\verb|PAPI_L2_ICM| & Level 2 instruction cache misses \\
\hline
\verb|PAPI_L2_TCM| & Level 2 cache misses \\
\hline
\verb|PAPI_LD_INS| & Load instructions \\
\hline
\verb|PAPI_SR_INS| & Store instructions \\
\hline
\verb|PAPI_TLB_DM| & Data translation lookaside buffer misses \\
\hline
\verb|PAPI_TOT_CYC| & Total cycles \\
\hline
\verb|PAPI_TOT_IIS| & Instructions issued \\
\hline
\verb|PAPI_TOT_INS| & Instructions completed \\
\hline
\end{tabular}
\end{center}
\caption{Some commonly available PAPI events.
The exact set of available events is system dependent.}
\label{tab:papi-events}
\end{table}

When selecting the period for PAPI events, aim for a rate of
approximately a few hundred samples per second.  So, roughly several
million or tens of million for total cycles or a few hundred thousand
for cache misses.  PAPI and \hpcrun{} will tolerate sampling rates as
high as 1,000 or even 10,000 samples per second (or more).  But rates
higher than a few thousand samples per second will only drive up the
overhead and distort the running of your program.  It won't give more
accurate results.

Earlier versions of PAPI required a separate patch (perfmon or
perfctr) for the Linux kernel.  But beginning with kernel 2.6.32,
support for accessing the performance counters (perf events) is now
built in to the standard Linux kernel.  This means that on kernels
2.6.32 or later, PAPI can be compiled and run entirely in user land
without patching the kernel.  PAPI is highly recommended and well
worth building if it is not already installed on your system.

\paragraph{Proxy Sampling}

\HPCToolkit{} now supports proxy sampling for derived PAPI events.
Normally, for \HPCToolkit{} to use a PAPI event, the event must not be
derived and must support hardware interrupts.  However, for events
that cannot trigger interrupts directly, it is still possible to
sample on another event and then read the counters for the derived
events and this is how proxy sampling works.  The native events serve
as a proxy for the derived events.

To use proxy sampling, specify the \hpcrun{} command line as usual and
be sure to include at least one non-derived PAPI event.  The derived
events will be counted automatically during the native samples.
Normally, you would use \verb|PAPI_TOT_CYC| as the native event, but
really this works as long as the event set contains at least one
non-derived PAPI event.  Proxy sampling only applies to PAPI events,
you can't use itimer as the native event.

For example, on newer Intel CPUs, often the floating point events are
all derived and cannot be sampled directly.  In that case, you could
count flops by using cycles a proxy event with a command line such as
the following.  The period for derived events is ignored and may be
omitted.

\begin{quote}
\begin{verbatim}
hpcrun -e PAPI_TOT_CYC@6000000 -e PAPI_FP_OPS app arg ...
\end{verbatim}
\end{quote}

Attribution of proxy samples is not as accurate as regular samples.
The problem, of course, is that the event that triggered the sample
may not be related to the derived counter.  The total count of events
should be accurate, but their location at the leaves in the Calling
Context tree may not be very accurate.  However, the higher up the
CCT, the more accurate the attribution becomes.  For example, suppose
you profile a loop of mixed integer and floating point operations and
sample on \verb|PAPI_TOT_CYC| directly and count \verb|PAPI_FP_OPS|
via proxy sampling.  The attribution of flops to individual statements
within the loop is likely to be off.  But as long as the loop is long
enough, the count for the loop as a whole (and up the tree) should be
accurate.

\subsection{Wallclock, Realtime and Cputime}

\HPCToolkit{} supports three timer sample sources: \verb|WALLCLOCK|,
\verb|REALTIME| and \verb|CPUTIME|.
The \verb|WALLCLOCK| sample source is based on the \verb|ITIMER_PROF|
interval timer.  Normally, \verb|PAPI_TOT_CYC| is just as good as
\verb|WALLCLOCK| and often better, but \verb|WALLCLOCK| can be used on
systems where PAPI is not available.  The units are in microseconds,
so the following example will sample \verb|app| approximately 200
times per second.

\begin{quote}
\begin{verbatim}
hpcrun -e WALLCLOCK@5000 app arg ...
\end{verbatim}
\end{quote}

Note that the maximum interrupt rate from itimer is limited by the
system's Hz rate, commonly 1,000 cycles per second, but may be lower.
That is, \verb|WALLCLOCK@10| will not generate any higher sampling rate
than \verb|WALLCLOCK@1000|.  However, on IBM Blue Gene, itimer is not
bound by the Hz rate and so sampling rates faster than 1,000 per
second are possible.

Also, the \verb|WALLCLOCK| (itimer) signal is not thread-specific and
may not work well in threaded programs.  In particular, the number of
samples per thread may vary wildly, although this is very system
dependent.  We recommend not using \verb|WALLCLOCK| in threaded
programs, except possibly on Blue Gene.  Use \verb|REALTIME|,
\verb|CPUTIME| or \verb|PAPI_TOT_CYC| instead.

The \verb|REALTIME| and \verb|CPUTIME| sources are based on the POSIX
timers \verb|CLOCK_REALTIME| and \verb|CLOCK_THREAD_CPUTIME_ID| with
the Linux \verb|SIGEV_THREAD_ID| extension.  \verb|REALTIME| counts
real (wall clock) time, whether the process is running or not, and
\verb|CPUTIME| only counts time when the CPU is running.  Both units
are in microseconds.

\verb|REALTIME| and \verb|CPUTIME| are not available on all systems
(in particular, not on Blue Gene), but they have important advantages
over itimer.  These timers are thread-specific and will give a much
more consistent number of samples in a threaded process.  Also,
compared to itimer, \verb|REALTIME| includes time when the process is
not running and so can identify times when the process is blocked
waiting on a syscall.  However, \verb|REALTIME| could also break some
applications that don't handle interrupted syscalls well.  In that
case, consider using \verb|CPUTIME| instead.

{\it Note:} do not use more than one timer event in the same run.
Also, we recommend not using both PAPI and a timer event together.
Technically, this is now possible and \hpcrun{} won't fall over.
However, PAPI samples would be interrupted by timer signals and vice
versa, and this would lead to many dropped samples and possibly
distorted results.

\subsection{IO}

The \verb|IO| sample source counts the number of bytes read and
written.  This displays two metrics in the viewer: ``IO Bytes Read''
and ``IO Bytes Written.''  The \verb|IO| source is a synchronous sample
source.  That is, it does not generate asynchronous interrupts.
Instead, it overrides the functions \verb|read|, \verb|write|, \verb|fread|
and \verb|fwrite| and records the number of bytes read or
written along with their dynamic context.

To include this source, use the \verb|IO| event (no period).  In the
static case, two steps are needed.  Use the \verb|--io| option for
\hpclink{} to link in the \verb|IO| library and use the \verb|IO| event
to activate the \verb|IO| source at runtime.  For example,

\begin{quote}
\begin{tabular}{@{}cl}
(dynamic) & \verb|hpcrun -e IO app arg ...| \\
(static)  & \verb|hpclink --io gcc -g -O -static -o app file.c ...| \\
& \verb|export HPCRUN_EVENT_LIST=IO| \\
& \verb|app arg ...|
\end{tabular}
\end{quote}

The \verb|IO| source is mainly used to find where your program reads or
writes large amounts of data.  However, it is also useful for tracing
a program that spends much time in \verb|read| and \verb|write|.  The
hardware performance counters (PAPI) do not advance while running in
the kernel, so the trace viewer may misrepresent the amount of time
spent in syscalls such as \verb|read| and \verb|write|.  By adding the
\verb|IO| source, \hpcrun{} overrides \verb|read| and \verb|write| and
thus is able to more accurately count the time spent in these
functions.

\subsection{Memleak}

The \verb|MEMLEAK| sample source counts the number of bytes allocated
and freed.  Like \verb|IO|, \verb|MEMLEAK| is a synchronous sample
source and does not generate asynchronous interrupts.  Instead, it
overrides the malloc family of functions (\verb|malloc|, \verb|calloc|,
\verb|realloc| and \verb|free| plus \verb|memalign|, \verb|posix_memalign|
and \verb|valloc|) and records the number of bytes
allocated and freed along with their dynamic context.

\verb|MEMLEAK| allows you to find locations in your program that
allocate memory that is never freed.  But note that failure to free a
memory location does not necessarily imply that location has leaked
(missing a pointer to the memory).  It is common for programs to
allocate memory that is used throughout the lifetime of the process
and not explicitly free it.

To include this source, use the \verb|MEMLEAK| event (no period).
Again, two steps are needed in the static case.  Use the \verb|--memleak|
option for \hpclink{} to link in the \verb|MEMLEAK| library
and use the \verb|MEMLEAK| event to activate it at runtime.  For
example,

\begin{quote}
\begin{tabular}{@{}cl}
(dynamic) & \verb|hpcrun -e MEMLEAK app arg ...| \\
(static)  & \verb|hpclink --memleak gcc -g -O -static -o app file.c ...| \\
& \verb|export HPCRUN_EVENT_LIST=MEMLEAK| \\
& \verb|app arg ...|
\end{tabular}
\end{quote}

If a program allocates and frees many small regions, the \verb|MEMLEAK|
source may result in a high overhead.  In this case, you may reduce
the overhead by using the memleak probability option to record only a
fraction of the mallocs.  For example, to monitor 10\% of the mallocs,
use:

\begin{quote}
\begin{tabular}{@{}cl}
(dynamic) & \verb|hpcrun -e MEMLEAK --memleak-prob 0.10 app arg ...| \\
(static)  & \verb|export HPCRUN_EVENT_LIST=MEMLEAK| \\
& \verb|export HPCRUN_MEMLEAK_PROB=0.10| \\
& \verb|app arg ...|
\end{tabular}
\end{quote}

It might appear that if you monitor only 10\% of the program's
mallocs, then you would have only a 10\% chance of finding the leak.
But if a program leaks memory, then it's likely that it does so many
times, all from the same source location.  And you only have to find
that location once.  So, this option can be a useful tool if the
overhead of recording all mallocs is prohibitive.

% ===========================================================================

\section{Process Fraction}

Although \hpcrun{} can profile parallel jobs with thousands or tens of
thousands of processes, there are two scaling problems that become
prohibitive beyond a few thousand cores.  First, \hpcrun{} writes the
measurement data for all of the processes into a single directory.
This results in one file per process plus one file per thread (two
files per thread if using tracing).  Unix file systems are not
equipped to handle directories with many tens or hundreds of thousands
of files.  Second, the sheer volume of data can overwhelm the viewer
when the size of the database far exceeds the amount of memory on the
machine.

The solution is to sample only a fraction of the processes.  That is,
you can run an application on many thousands of cores but record data
for only a few hundred processes.  The other processes run the
application but do not record any measurement data.  This is what the
process fraction option (\verb|-f| or \verb|--process-fraction|) does.
For example, to monitor 10\% of the processes, use:

\begin{quote}
\begin{tabular}{@{}cl}
(dynamic) & \verb|hpcrun -f 0.10 -e event@period app arg ...| \\
(dynamic) & \verb|hpcrun -f 1/10 -e event@period app arg ...| \\
(static)  & \verb|export HPCRUN_EVENT_LIST='event@period'| \\
& \verb|export HPCRUN_PROCESS_FRACTION=0.10| \\
& \verb|app arg ...|
\end{tabular}
\end{quote}

With this option, each process generates a random number and records
its measurement data with the given probability.  The process fraction
(probability) may be written as a decimal number (0.10) or as a
fraction (1/10) between 0 and 1.  So, in the above example, all three
cases would record data for approximately 10\% of the processes.  Aim
for a number of processes in the hundreds.

% ===========================================================================

\section{Starting and Stopping Sampling}

\HPCToolkit{} supports an API for the application to start and stop
sampling.  This is useful if you want to profile only a subset of a
program and ignore the rest.  The API supports the following
functions.

\begin{quote}
\begin{verbatim}
void hpctoolkit_sampling_start(void);
void hpctoolkit_sampling_stop(void);
\end{verbatim}
\end{quote}

For example, suppose that your program has three major phases: it
reads input from a file, performs some numerical computation on the
data and then writes the output to another file.  And suppose that you
want to profile only the compute phase and skip the read and write
phases.  In that case, you could stop sampling at the beginning of the
program, restart it before the compute phase and stop it again at the
end of the compute phase.

This interface is process wide, not thread specific.  That is, it
affects all threads of a process.  Note that when you turn sampling on
or off, you should do so uniformly across all processes, normally at
the same point in the program.  Enabling sampling in only a subset of
the processes would likely produce skewed and misleading results.

And for technical reasons, when sampling is turned off in a threaded
process, interrupts are disabled only for the current thread.  Other
threads continue to receive interrupts, but they don't unwind the call
stack or record samples.  So, another use for this interface is to
protect syscalls that are sensitive to being interrupted with signals.
For example, some Gemini interconnect (GNI) functions called from
inside \verb|gasnet_init()| or \verb|MPI_Init()| on Cray XE systems
will fail if they are interrupted by a signal.  As a workaround, you
could turn sampling off around those functions.

Also, you should use this interface only at the top level for major
phases of your program.  That is, the granularity of turning sampling
on and off should be much larger than the time between samples.
Turning sampling on and off down inside an inner loop will likely
produce skewed and misleading results.

To use this interface, put the above function calls into your program
where you want sampling to start and stop.  Remember, starting and
stopping apply process wide.  For C/C++, include the following header
file from the \HPCToolkit{} \verb|include| directory.

\begin{quote}
\begin{verbatim}
#include <hpctoolkit.h>
\end{verbatim}
\end{quote}

Compile your application with \verb|libhpctoolkit| with \verb|-I| and
\verb|-L| options for the include and library paths.  For example,

\begin{quote}
\begin{verbatim}
gcc -I /path/to/hpctoolkit/include app.c ... \
    -L /path/to/hpctoolkit/lib/hpctoolkit -lhpctoolkit ...
\end{verbatim}
\end{quote}

The \verb|libhpctoolkit| library provides weak symbol no-op definitions
for the start and stop functions.  For dynamically linked programs, be
sure to include \verb|-lhpctoolkit| on the link line (otherwise your
program won't link).  For statically linked programs, \hpclink{} adds
strong symbol definitions for these functions.  So, \verb|-lhpctoolkit|
is not necessary in the static case, but it doesn't hurt.

To run the program, set the \verb|LD_LIBRARY_PATH| environment
variable to include the \HPCToolkit{} \verb|lib/hpctoolkit| directory.
This step is only needed for dynamically linked programs.

\begin{quote}
\begin{verbatim}
export LD_LIBRARY_PATH=/path/to/hpctoolkit/lib/hpctoolkit
\end{verbatim}
\end{quote}

Note that sampling is initially turned on until the program turns it
off.  If you want it initially turned off, then use the \verb|-ds| (or
\verb|--delay-sampling|) option for \hpcrun{} (dynamic) or set the
\verb|HPCRUN_DELAY_SAMPLING| environment variable (static).

\begin{quote}
\begin{tabular}{@{}cl}
(dynamic) & \verb|hpcrun -ds -e event@period app arg ...|  \\
(static)  & \verb|export HPCRUN_EVENT_LIST='event@period'| \\
& \verb|export HPCRUN_DELAY_SAMPLING=1| \\
& \verb|app arg ...|
\end{tabular}
\end{quote}

% ===========================================================================
\section{Environment Variables for \hpcrun{}}
\label{sec:env-vars}
For most systems, \hpcrun{} requires no special environment variable settings.
There are two situations, however, where \hpcrun{}, to function correctly,
\emph{must} refer to environment variables. These environment variables, and
corresponding situations are:
\begin{description}
  \item{\verb|HPCTOOLKIT|} To function correctly, \hpcrun{} must know
       the location of the \HPCToolkit{} top-level installation directory.
       The \hpcrun{} script uses elements of the installation \verb|lib| and
       \verb|libexec| subdirectories. On most systems, the 
       \hpcrun{} can find the requisite
       components relative to its own location in the file system. 
       However, some parallel job launchers \emph{copy} the
       \hpcrun{} script to a different location as they launch a job. If your
       system does this, you must set the \verb|HPCTOOLKIT|
       environment variable to the location of the \HPCToolkit{} top-level installation directory 
       before launching a job.
  \item{\verb|HPCRUN_TMPDIR|} While profiling a dynamically-linked executable, \hpcrun{} needs
       to write some internal data to temporary files. By default, these files
       are written to \verb|/tmp|. However, on some platforms (e.g, a Cray XK6), you may not
       have permission to write to \verb|/tmp|. If this is the case on your system,
       you should set the \verb|HPCRUN_TMPDIR| environment variable to a directory
       where you have write permission. If your system has a shared scratch parallel file system, 
       setting \verb|HPCRUN_TMPDIR| to point to a subdirectory of your's there would be a good choice.

       NOTE: You do not strictly have to use \verb|HPCRUN_TMPDIR|.  If \verb|HPCRUN_TMPDIR| is not set, 
       \hpcrun{} will place its temporary files in a directory pointed to by 
       the commonly-used Unix environment variable \verb|TMPDIR|. If  \verb|TMPDIR| is not set either,
       then HPCToolkit will attempt to write its temporary files in /tmp.
\end{description}

{\bf Note to system administrators:} if your system provides a module system for configuring 
software packages, then constructing
a module for \HPCToolkit{} to initialize these environment variables to appropriate settings
would be convenient for users.

\section{Platform-Specific Notes}

%
% system specific notes for titan, & tmpdir ? keenland?
%
\subsection{Cray XE6 and XK6}

The ALPS job launcher used on Cray XE6 and XK6 systems copies 
programs to a special staging area before launching them,
as described in Section~\ref{sec:env-vars}.
Consequently, when using \hpcrun{} to monitor dynamically linked binaries on Cray XE6 and XK6 systems, you
should add the \verb|HPCTOOLKIT| environment variable to your launch
script.  
Set \verb|HPCTOOLKIT| to the top-level \HPCToolkit{} installation directory
(the directory containing the \verb|bin|, \verb|lib| and
\verb|libexec| subdirectories) and export it to the environment.  
(If launching statically-linked binaries created using \hpclink{}, this step is unnecessary, but harmless.)
Below we show a skeletal job script that sets the \verb|HPCTOOLKIT| environment variable  before monitoring 
a dynamically-linked executable with \hpcrun{}: 

\begin{quote}
\begin{verbatim}
#!/bin/sh
#PBS -l mppwidth=#nodes
#PBS -l walltime=00:30:00
#PBS -V

export HPCTOOLKIT=/path/to/hpctoolkit/install/directory
export CRAY_ROOTFS=DSL

cd $PBS_O_WORKDIR
aprun -n #nodes hpcrun -e event@period dynamic-app arg ...
\end{verbatim}
\end{quote}
% $

If \verb|HPCTOOLKIT| is not set, you may see errors such as the
following in your job's error log.

\begin{quote}
\begin{verbatim}
/var/spool/alps/103526/hpcrun: Unable to find HPCTOOLKIT root directory.
Please set HPCTOOLKIT to the install prefix, either in this script,
or in your environment, and try again.
\end{verbatim}
\end{quote}

The problem is that the Cray job launcher copies the \hpcrun{}
script to a directory somewhere below \verb|/var/spool/alps/|\ and runs
it from there.  By moving \hpcrun{} to a different directory, this
breaks \hpcrun{}'s method for finding its own install directory.
The solution is to add \verb|HPCTOOLKIT| to your environment so that
\hpcrun{} can find its install directory.

Your system may have a module installed for \verb|hpctoolkit| with the
correct settings for \verb|PATH|, \verb|HPCTOOLKIT|, etc.  In that case,
the easiest solution is to load the \verb|hpctoolkit| module.  Try
``\verb|module show hpctoolkit|'' to see if it sets \verb|HPCTOOLKIT|.
