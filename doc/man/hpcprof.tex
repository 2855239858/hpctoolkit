%% $Id$

%%%%%%%%%%%%%%%%%%%%%%%%%%%%%%%%%%%%%%%%%%%%%%%%%%%%%%%%%%%%%%%%%%%%%%%%%%%%%
%%%%%%%%%%%%%%%%%%%%%%%%%%%%%%%%%%%%%%%%%%%%%%%%%%%%%%%%%%%%%%%%%%%%%%%%%%%%%

\documentclass[english]{article}
\usepackage[latin1]{inputenc}
\usepackage{babel}
\usepackage{verbatim}

%% do we have the `hyperref package?
\IfFileExists{hyperref.sty}{
   \usepackage[bookmarksopen,bookmarksnumbered]{hyperref}
}{}

%% do we have the `fancyhdr' or `fancyheadings' package?
\IfFileExists{fancyhdr.sty}{
\usepackage[fancyhdr]{latex2man}
}{
\IfFileExists{fancyheadings.sty}{
\usepackage[fancy]{latex2man}
}{
\usepackage[nofancy]{latex2man}
\message{no fancyhdr or fancyheadings package present, discard it}
}}

%% do we have the `rcsinfo' package?
\IfFileExists{rcsinfo.sty}{
\usepackage[nofancy]{rcsinfo}
\rcsInfo $Id$
\setDate{\rcsInfoLongDate}
}{
\setDate{ 2007/06/04}
\message{package rcsinfo not present, discard it}
}

\setVersionWord{Version:}  %%% that's the default, no need to set it.
\setVersion{=PACKAGE_VERSION=}

%%%%%%%%%%%%%%%%%%%%%%%%%%%%%%%%%%%%%%%%%%%%%%%%%%%%%%%%%%%%%%%%%%%%%%%%%%%%%
%%%%%%%%%%%%%%%%%%%%%%%%%%%%%%%%%%%%%%%%%%%%%%%%%%%%%%%%%%%%%%%%%%%%%%%%%%%%%

\begin{document}

\begin{Name}{1}{hpcprof}{The HPCToolkit Performance Tools}{The HPCToolkit Performance Tools}{hpcprof:\\ Simple correlation of flat profiles with source code}

\Prog{hpcprof} correlates `flat' execution profiles with source code procedures and lines and generates plain-text or HTML output.
It is based on Curtis Janssen's VProf.

\end{Name}

%%%%%%%%%%%%%%%%%%%%%%%%%%%%%%%%%%%%%%%%%%%%%%%%%%%%%%%%%%%%%%%%%%
\section{Synopsis}

\Prog{hpcprof} \oOpt{options} \Arg{binary} \Arg{hpcrun-file}...

%%%%%%%%%%%%%%%%%%%%%%%%%%%%%%%%%%%%%%%%%%%%%%%%%%%%%%%%%%%%%%%%%%
\section{Description}

\Prog{hpcprof} correlates `flat' execution profiles of \Arg{binary} with source code files, procedures, lines or with object code.
It is typically used to generate plain-text or HTML output.
\Arg{hpcrun-file} is a collection of event-based program counter histograms obtained using \Cmd{hpcrun}{1} (or \Cmd{hpcex}{1}).

\Prog{hpcprof} can generate the following information.  Note that the term \emph{native} event refers to an event for which there is a program counter histogram.  
\begin{Description}
  \item[totals] Total sample counts for each native event.  If the same native event name appears more than once, min/max/sum \emph{derived} events are computed.
  \item[load-module correlation] For each \emph{native} event (in order), show the exclusive percentage (or number) of samples attributed to each load module in the \Arg{hpcrun-file}s.
  \item[file correlation] For each \emph{native} event (in order), show the exclusive percentage (or number) of samples attributed to each $<$load module$>$$<$source file$>$ pair.
  \item[function correlation] For each \emph{native} event (in order), show the exclusive percentage (or number) of samples attributed to each $<$load module$>$$<$function$>$ pair.
  \item[line correlation] For each \emph{native} event (in order), show the exclusive percentage (or number) of samples attributed to each $<$load module$>$$<$file$>$$<$line$>$ triple.
  \item[object code correlation] For each \emph{native} event (in order), show the exclusive percentage (or \emph{number}) of samples attributed to each (text segment) object code instruction from each load module.  NOTE: On ISA's with variable sized instructions, histogram buckets (4 bytes in size) may contain information for more than one instruction.  In this case multiple instructions will report counts for the \emph{same} bucket.
\end{Description}

%%%%%%%%%%%%%%%%%%%%%%%%%%%%%%%%%%%%%%%%%%%%%%%%%%%%%%%%%%%%%%%%%%
\section{Arguments}

\begin{Description}
\item[\Arg{binary}] An executable binary.
\item[\Arg{hpcrun-file}...] A non-empty list of \Arg{hpcrun-file}s.
\end{Description}

Default values for an option's optional arguments are shown in \{\}.

\subsection{Options: General}

\begin{Description}
\item[\OptArg{-d}{dir}, \OptArg{--directory}{dir}] Search \Arg{dir} for source files.
\item[\OptArg{-D}{dir}, \OptArg{--recursive-directory}{dir}] Search \Arg{dir} recursively for source files.
\item[\Opt{--force}] Show data that is not accurate.
\item[\Opt{-V}, \Opt{--version}] Print version information.
\item[\Opt{-h}, \Opt{--help}] Print help.
\item[\OptArg{--debug}{n}]   Debug: use debug level \Arg{n}.
\end{Description}

\subsection{Options: Plain-text and HTML Output}
\begin{Description}
  \item[\Opt{-e}, \Opt{--everything}] load-module, file, function and line correlations.
  \item[\Opt{-f}, \Opt{--files}] Show file correlation described above.
  \item[\Opt{-r}, \Opt{--funcs}] Show function correlation described above.
  \item[\Opt{-l}, \Opt{--lines}] Show line correlation described above.

  \item[\Opt{-o}, \Opt{--object}] Show object code correlation (plain-text only).  Use the \Opt{-l}, \Opt{--lines} option to intermingle source line information with object code.

  \item[\Opt{-n}, \Opt{--number}] Show number of samples instead of percentage
  \item[\OptArg{-s}{n}, \OptArg{--show}{n}] Use threshold \Arg{n} for showing aggregate data

  \item[\OptArg{-H}{dir}, \OptArg{--html}{dir}] Output HTML into directory \Arg{dir}
  \item[\OptArg{-a}{file}, \OptArg{--annotate}{file}] Annotate the source file \Arg{file}
\end{Description}

\subsection{Options: PROFILE Output}
\begin{Description}
  \item[\Opt{-p}, \Opt{--profile}] Generate PROFILE output to stdout. Note: This option is deprecated; use \Cmd{hpcquick}{1}/\Cmd{hpcview}{1} for improved correlation.
\end{Description}



%%%%%%%%%%%%%%%%%%%%%%%%%%%%%%%%%%%%%%%%%%%%%%%%%%%%%%%%%%%%%%%%%%
\section{Examples}

Assume we have collected flat profiling information using \Cmd{hpcrun}{1} \Cmd{hpcex}{1}).  Let the profile be named \File{}

%\begin{enumerate}
%\item 

 and that we wish to correlate the profile with source lines.
We wish to recover program structure in the file \File{sweep3dsingle.psxml} for use with \Cmd{hpcquick}{1}/\Cmd{hpcview}{1}.
To do this, execute:
\begin{verbatim}
    bloop sweep3dsingle > sweep3dsingle.psxml
\end{verbatim}

%\end{enumerate}

%%%%%%%%%%%%%%%%%%%%%%%%%%%%%%%%%%%%%%%%%%%%%%%%%%%%%%%%%%%%%%%%%%
%\section{Notes}

%%%%%%%%%%%%%%%%%%%%%%%%%%%%%%%%%%%%%%%%%%%%%%%%%%%%%%%%%%%%%%%%%%
\section{See Also}

\Cmd{hpctoolkit}{1}.

%%%%%%%%%%%%%%%%%%%%%%%%%%%%%%%%%%%%%%%%%%%%%%%%%%%%%%%%%%%%%%%%%%
\section{Version}

Version: \Version\ of \Date.

%%%%%%%%%%%%%%%%%%%%%%%%%%%%%%%%%%%%%%%%%%%%%%%%%%%%%%%%%%%%%%%%%%
\section{License and Copyright}

\begin{description}
\item[Copyright] \copyright\ 2002-2007, Rice University.
\item[License] See \File{README.License}.
\end{description}

%%%%%%%%%%%%%%%%%%%%%%%%%%%%%%%%%%%%%%%%%%%%%%%%%%%%%%%%%%%%%%%%%%
\section{Authors}

\noindent
Nathan Tallent \\
John Mellor-Crummey \\
Rob Fowler \\
Email: \Email{hpc@cs.rice.edu} \\
WWW: \URL{http://hipersoft.cs.rice.edu/hpctoolkit}.

\LatexManEnd

\end{document}

%% Local Variables:
%% eval: (add-hook 'write-file-hooks 'time-stamp)
%% time-stamp-start: "setDate{ "
%% time-stamp-format: "%:y/%02m/%02d"
%% time-stamp-end: "}\n"
%% time-stamp-line-limit: 50
%% End:

