%% $Id$

%%%%%%%%%%%%%%%%%%%%%%%%%%%%%%%%%%%%%%%%%%%%%%%%%%%%%%%%%%%%%%%%%%%%%%%%%%%%%
%%%%%%%%%%%%%%%%%%%%%%%%%%%%%%%%%%%%%%%%%%%%%%%%%%%%%%%%%%%%%%%%%%%%%%%%%%%%%

\documentclass[english]{article}
\usepackage[latin1]{inputenc}
\usepackage{babel}
\usepackage{verbatim}

%% do we have the `hyperref package?
\IfFileExists{hyperref.sty}{
   \usepackage[bookmarksopen,bookmarksnumbered]{hyperref}
}{}

%% do we have the `fancyhdr' or `fancyheadings' package?
\IfFileExists{fancyhdr.sty}{
\usepackage[fancyhdr]{latex2man}
}{
\IfFileExists{fancyheadings.sty}{
\usepackage[fancy]{latex2man}
}{
\usepackage[nofancy]{latex2man}
\message{no fancyhdr or fancyheadings package present, discard it}
}}

%% do we have the `rcsinfo' package?
\IfFileExists{rcsinfo.sty}{
\usepackage[nofancy]{rcsinfo}
\rcsInfo $Id$
\setDate{\rcsInfoLongDate}
}{
\setDate{ 2012/04/18}
\message{package rcsinfo not present, discard it}
}

\setVersionWord{Version:}  %%% that's the default, no need to set it.
\setVersion{=PACKAGE_VERSION=}

%%%%%%%%%%%%%%%%%%%%%%%%%%%%%%%%%%%%%%%%%%%%%%%%%%%%%%%%%%%%%%%%%%%%%%%%%%%%%
%%%%%%%%%%%%%%%%%%%%%%%%%%%%%%%%%%%%%%%%%%%%%%%%%%%%%%%%%%%%%%%%%%%%%%%%%%%%%

\begin{document}

\begin{Name}{1}{hpcprof-mpi}{The HPCToolkit Performance Tools}{The HPCToolkit Performance Tools}{hpcprof-mpi:\\ Analysis and Attribution of Call Path Performance Measurements}

\Prog{hpcprof-mpi} \emph{scalably} analyzes call path profile performance measurements (in parallel) and attributes them to static source code structure.
See \HTMLhref{hpctoolkit.html}{\Cmd{hpctoolkit}{1}} for an overview of \textbf{HPCToolkit}.

\end{Name}

%%%%%%%%%%%%%%%%%%%%%%%%%%%%%%%%%%%%%%%%%%%%%%%%%%%%%%%%%%%%%%%%%%
\section{Synopsis}

\Prog{hpcprof-mpi} \oOpt{options} \Arg{measurement-group}...

%%%%%%%%%%%%%%%%%%%%%%%%%%%%%%%%%%%%%%%%%%%%%%%%%%%%%%%%%%%%%%%%%%
\section{Description}

\Prog{hpcprof-mpi} analyzes call path profile performance measurements, attributes them to static source code structure, and generates an Experiment database for use with \HTMLhref{hpcviewer.html}{\Cmd{hpcviewer}{1}}.
\Prog{hpcprof-mpi} is especially designed for analyzing and attributing measurements from large-scale executions.

\Prog{hpcprof-mpi} expects a list of measurement-groups, where a group is a call path profile directory or an individual profile file.
For best results, two other options should be used: \textbf{-I} to provide search directories for source code and \textbf{-S} to provide source code structure from \HTMLhref{hpcstruct.html}{\Cmd{hpcstruct}{1}}.
(Note that without any search-directory argument, \Prog{hpcprof} will only find source files that either (1) have absolute paths (and that still exist on the file system) or (2) are relative to the current working directory.)


%%%%%%%%%%%%%%%%%%%%%%%%%%%%%%%%%%%%%%%%%%%%%%%%%%%%%%%%%%%%%%%%%%
\section{Arguments}

\begin{Description}
\item[\Arg{measurement-group}...] A list of measurement-groups, where a group is a call path profile directory or an individual profile file.
\end{Description}

Default values for an option's optional arguments are shown in \{\}.

\subsection{Options: Informational}

\begin{Description}
\item[\OptoArg{-v}{n}, \OptoArg{--verbose}{n}]
Verbose: generate progress messages to stderr at verbosity level \Arg{n}.  \{1\} 
\item[\Opt{-V}, \Opt{--version}]
Print version information.

\item[\Opt{-h}, \Opt{--help}]
Print help.

\item[\OptoArg{--debug}{n}]
Debug: use debug level \Arg{n}. \{1\}
\end{Description}

\subsection{Options: Source Code and Static Structure}

\begin{Description}
\item[\OptArg{--name}{name}, \OptArg{--title}{name}]
Set the database's name (title) to \Arg{name}.

\item[\OptArg{-I}{dir}, \OptArg{--include}{dir}]
Use \Arg{dir} as a search directory to find source files.
For a recursive search, append a '*' after the last slash, e.g., \verb+'/mypath/*'+ (quote or escape to protect from the shell).
May pass multiple times.

Note: With multiple search-directory arguments, it may be the case that file \Arg{f} exists within more than one search directory.
In this case, the ambiguity is resolved in favor of the search directory that appears first on the command line.

\item[\OptArg{-S}{file}, \OptArg{--structure}{file}]
Use \HTMLhref{hpcstruct.html}{\Cmd{hpcstruct}{1}} structure file \Arg{file} for correlation.
May pass multiple times (e.g., for shared libraries).

\item[\OptArg{-R}{'old-path=new-path'}, \OptArg{--replace-path}{'old-path=new-path'}]
Substitute instances of \Arg{old-path} with \Arg{new-path}; apply to all paths (e.g., a profile's load map and source code) for which \Arg{old-path} is a prefix.
Use '\Bs'\ to escape instances of '=' within a path. May pass multiple times.
  
Use this option when a profile or binary contains references to files that have been relocated, such as might occur with a file system change.
\end{Description}

\subsection{Options: Metrics}

\begin{Description}

\item[\OptArg{-M}{metric}, \OptArg{--metric}{metric}]
Specify the set of metrics to compute, where \Arg{metric} is one of the following:
  \begin{itemize}
  \item[sum] Sum over threads/processes
  \item[stats] Sum, Mean, StdDev (standard deviation), CoefVar (coefficient of variation), Min, Max over threads/processes
  \item[thread] per-thread/process metrics
  \end{itemize}
Default is 'sum'.
May pass multiple times.
(\Cmd{hpcprof-mpi}{1} does not compute 'thread'.)

\end{Description}

\subsection{Options: Output}

\begin{Description}

\item[\OptArg{-o}{db-path}, \OptArg{--db}{db-path}, \OptArg{--output}{db-path}]
Specify Experiment database name \Arg{db-path}.
\{\File{./hpctoolkit-$<$application$>$-database}\}

\item[\OptArg{--metric-db}{yes|no}]
Control whether to generate a thread-level metric value database for \HTMLhref{hpcviewer.html}{\Cmd{hpcviewer}{1}} scatter plots.
\{yes\}

\end{Description}


%%%%%%%%%%%%%%%%%%%%%%%%%%%%%%%%%%%%%%%%%%%%%%%%%%%%%%%%%%%%%%%%%%
\section{Examples}

%\begin{enumerate}
%\item 


%\end{enumerate}

%%%%%%%%%%%%%%%%%%%%%%%%%%%%%%%%%%%%%%%%%%%%%%%%%%%%%%%%%%%%%%%%%%
%\section{Notes}


%%%%%%%%%%%%%%%%%%%%%%%%%%%%%%%%%%%%%%%%%%%%%%%%%%%%%%%%%%%%%%%%%%
\section{See Also}

\HTMLhref{hpctoolkit.html}{\Cmd{hpctoolkit}{1}}.\\
\HTMLhref{hpcprof.html}{\Cmd{hpcprof}{1}}.\\
\HTMLhref{hpcproftt.html}{\Cmd{hpcproftt}{1}}.\\
\HTMLhref{hpcsummary.html}{\Cmd{hpcsummary}{1}}.

%%%%%%%%%%%%%%%%%%%%%%%%%%%%%%%%%%%%%%%%%%%%%%%%%%%%%%%%%%%%%%%%%%
\section{Version}

Version: \Version\ of \Date.

%%%%%%%%%%%%%%%%%%%%%%%%%%%%%%%%%%%%%%%%%%%%%%%%%%%%%%%%%%%%%%%%%%
\section{License and Copyright}

\begin{description}
\item[Copyright] \copyright\ 2002-2013, Rice University.
\item[License] See \File{README.License}.
\end{description}

%%%%%%%%%%%%%%%%%%%%%%%%%%%%%%%%%%%%%%%%%%%%%%%%%%%%%%%%%%%%%%%%%%
\section{Authors}

\noindent
Nathan Tallent \\
John Mellor-Crummey \\
Rice HPCToolkit Research Group \\
Email: \Email{hpctoolkit-forum =at= rice.edu} \\
WWW: \URL{http://hpctoolkit.org}.

\LatexManEnd

\end{document}

%% Local Variables:
%% eval: (add-hook 'write-file-hooks 'time-stamp)
%% time-stamp-start: "setDate{ "
%% time-stamp-format: "%:y/%02m/%02d"
%% time-stamp-end: "}\n"
%% time-stamp-line-limit: 50
%% End:
