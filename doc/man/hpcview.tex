%% $Id$

%%%%%%%%%%%%%%%%%%%%%%%%%%%%%%%%%%%%%%%%%%%%%%%%%%%%%%%%%%%%%%%%%%%%%%%%%%%%%
%%%%%%%%%%%%%%%%%%%%%%%%%%%%%%%%%%%%%%%%%%%%%%%%%%%%%%%%%%%%%%%%%%%%%%%%%%%%%

\documentclass[english]{article}
\usepackage[latin1]{inputenc}
\usepackage{babel}
\usepackage{verbatim}

%% do we have the `hyperref package?
\IfFileExists{hyperref.sty}{
   \usepackage[bookmarksopen,bookmarksnumbered]{hyperref}
}{}

%% do we have the `fancyhdr' or `fancyheadings' package?
\IfFileExists{fancyhdr.sty}{
\usepackage[fancyhdr]{latex2man}
}{
\IfFileExists{fancyheadings.sty}{
\usepackage[fancy]{latex2man}
}{
\usepackage[nofancy]{latex2man}
\message{no fancyhdr or fancyheadings package present, discard it}
}}

%% do we have the `rcsinfo' package?
\IfFileExists{rcsinfo.sty}{
\usepackage[nofancy]{rcsinfo}
\rcsInfo $Id$
\setDate{\rcsInfoLongDate}
}{
\setDate{ 2007/05/25}
\message{package rcsinfo not present, discard it}
}

%\setVersionWord{Version:}  %%% that's the default, no need to set it.
\setVersion{@PACKAGE_VERSION@}

\setVersionWord{Version:}  %%% that's the default, no need to set it.
\setVersion{@PACKAGE_VERSION@}

%%%%%%%%%%%%%%%%%%%%%%%%%%%%%%%%%%%%%%%%%%%%%%%%%%%%%%%%%%%%%%%%%%%%%%%%%%%%%
%%%%%%%%%%%%%%%%%%%%%%%%%%%%%%%%%%%%%%%%%%%%%%%%%%%%%%%%%%%%%%%%%%%%%%%%%%%%%

\begin{document}

\begin{Name}{1}{hpcview}{The HPCToolkit Performance Tools}{The HPCToolkit Performance Tools}{hpcview:\\ Correlation of Flat Profiles with Static Program Structure}

\Prog{hpcview} correlates dynamic (`flat') profiles with static source code structure.  See \Cmd{hpctoolkit}{1} for an overview of \textbf{HPCToolkit}.

\end{Name}

%%%%%%%%%%%%%%%%%%%%%%%%%%%%%%%%%%%%%%%%%%%%%%%%%%%%%%%%%%%%%%%%%%
\section{Synopsis}

\Prog{hpcview} \oOpt{options} \Arg{config-file} \oOpt{profile-file...}

%%%%%%%%%%%%%%%%%%%%%%%%%%%%%%%%%%%%%%%%%%%%%%%%%%%%%%%%%%%%%%%%%%
\section{Description}

\Prog{hpcview} correlates dynamic profiling metrics with static source code structure and (by default) generates an Experiment database for use with \Cmd{hpcviewer}{1}.
\Prog{hpcview} is driven by the configuration file \Arg{config-file}, which amongh other things, may contain user defined derived metrics.
\Arg{profile-files} is a list of \Cmd{hpcrun}{1} profiles.


FIXME [config file]

%%%%%%%%%%%%%%%%%%%%%%%%%%%%%%%%%%%%%%%%%%%%%%%%%%%%%%%%%%%%%%%%%%
\section{Arguments}

\begin{Description}
\item[\Arg{config-file...}] The \Prog{hpcview} configuration file.
\item[\Arg{profile-file...}] \Cmd{hpcrun}{1} profile files.
\end{Description}

Default values for an option's optional arguments are shown in \{\}.

\subsection{Options: General}

\begin{Description}
\item[\OptoArg{-v}{n}, \OptoArg{--verbose}{n}] Verbose: generate progress messages to stderr at verbosity level \Arg{n}.  \{1\}  (Use n=2 to debug path replacement if metric and program structure is not properly matched.)
\item[\Opt{-V}, \Opt{--version}] Print version information.
\item[\Opt{-h}, \Opt{--help}] Print help.
\item[\OptoArg{--debug}{n}]   Debug: use debug level \Arg{n}. \{1\}
\end{Description}


\subsection{Options: Output}

\begin{Description}
  \item[\OptArg{-o}{db-path}, \OptArg{--db}{db-path}, \OptArg{--output}{db-path}] Specify Experiment database name \Arg{db-path}.  \{\File{./experiment-db}\}
  \item[\OptoArg{--src}{yes \Bar no}, \OptoArg{--source}{yes \Bar no}] Whether to copy source code files into Experiment database. \{yes\} By default, \Prog{hpcview} copies source files with performance metrics and that can be reached by PATH/REPLACE statements, resulting in a self-contained dataset that does not rely on an external source code repository.  Note that if copying is suppressed, the database is no longer self-contained.
\end{Description}

\subsection{Output Formats}

Select different output formats and optionally specify the output filename \Arg{fname} (located within the Experiment database). The output is sparse in the sense that it ignores program areas without profiling information. (Set \Arg{fname} to '-' to write to stdout.)

\begin{Description}
  \item[\OptoArg{-x}{fname}, \OptoArg{--experiment}{fname}] Default.  ExperimentXML format. \{\File{experiment.xml}\}.  NOTE: To disable, set \Arg{fname} to \verb+no+.
  \item[\OptoArg{--csv}{fname}] Comma-separated-value format. \{\File{experiment.csv}\}. Includes flat scope tree and loops.  Useful for downstream external tools.
  \item[\OptoArg{--tsv}{fname}] Tab-separated-value format. \{\File{experiment.tsv}\}. Includes flat scope tree and lines.  Useful for downstream external tools.
\end{Description}


%%%%%%%%%%%%%%%%%%%%%%%%%%%%%%%%%%%%%%%%%%%%%%%%%%%%%%%%%%%%%%%%%%
\section{Examples}

%\begin{enumerate}
%\item 
[FIXME]

%\end{enumerate}

%%%%%%%%%%%%%%%%%%%%%%%%%%%%%%%%%%%%%%%%%%%%%%%%%%%%%%%%%%%%%%%%%%
%\section{Notes}


%%%%%%%%%%%%%%%%%%%%%%%%%%%%%%%%%%%%%%%%%%%%%%%%%%%%%%%%%%%%%%%%%%
\section{See Also}

\Cmd{hpctoolkit}{1}.

%%%%%%%%%%%%%%%%%%%%%%%%%%%%%%%%%%%%%%%%%%%%%%%%%%%%%%%%%%%%%%%%%%
\section{Version}

Version: \Version\ of \Date.

%%%%%%%%%%%%%%%%%%%%%%%%%%%%%%%%%%%%%%%%%%%%%%%%%%%%%%%%%%%%%%%%%%
\section{License and Copyright}

\begin{description}
\item[Copyright] \copyright\ 2002-2007, Rice University.
\item[License] See \File{README.License}.
\end{description}

%%%%%%%%%%%%%%%%%%%%%%%%%%%%%%%%%%%%%%%%%%%%%%%%%%%%%%%%%%%%%%%%%%
\section{Authors}

\noindent
John Mellor-Crummey \\
Nathan Tallent \\
Rob Fowler \\
Email: \Email{hpc@cs.rice.edu} \\
WWW: \URL{http://hipersoft.cs.rice.edu/hpctoolkit}.

\LatexManEnd

\end{document}

%% Local Variables:
%% eval: (add-hook 'write-file-hooks 'time-stamp)
%% time-stamp-start: "setDate{ "
%% time-stamp-format: "%:y/%02m/%02d"
%% time-stamp-end: "}\n"
%% time-stamp-line-limit: 50
%% End:

