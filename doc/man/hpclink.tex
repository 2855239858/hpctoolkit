%% $Id$

%%%%%%%%%%%%%%%%%%%%%%%%%%%%%%%%%%%%%%%%%%%%%%%%%%%%%%%%%%%%%%%%%%%%%%%%%%%%%
%%%%%%%%%%%%%%%%%%%%%%%%%%%%%%%%%%%%%%%%%%%%%%%%%%%%%%%%%%%%%%%%%%%%%%%%%%%%%

\documentclass[english]{article}
\usepackage[latin1]{inputenc}
\usepackage{babel}
\usepackage{verbatim}

%% do we have the `hyperref package?
\IfFileExists{hyperref.sty}{
   \usepackage[bookmarksopen,bookmarksnumbered]{hyperref}
}{}

%% do we have the `fancyhdr' or `fancyheadings' package?
\IfFileExists{fancyhdr.sty}{
\usepackage[fancyhdr]{latex2man}
}{
\IfFileExists{fancyheadings.sty}{
\usepackage[fancy]{latex2man}
}{
\usepackage[nofancy]{latex2man}
\message{no fancyhdr or fancyheadings package present, discard it}
}}

%% do we have the `rcsinfo' package?
\IfFileExists{rcsinfo.sty}{
\usepackage[nofancy]{rcsinfo}
\rcsInfo $Id$
\setDate{\rcsInfoLongDate}
}{
\setDate{ 2009/06/21}
\message{package rcsinfo not present, discard it}
}

\setVersionWord{Version:}  %%% that's the default, no need to set it.
\setVersion{=PACKAGE_VERSION=}

%%%%%%%%%%%%%%%%%%%%%%%%%%%%%%%%%%%%%%%%%%%%%%%%%%%%%%%%%%%%%%%%%%%%%%%%%%%%%
%%%%%%%%%%%%%%%%%%%%%%%%%%%%%%%%%%%%%%%%%%%%%%%%%%%%%%%%%%%%%%%%%%%%%%%%%%%%%

\begin{document}

\begin{Name}{1}{hpclink}{The HPCToolkit Performance Tools}{The HPCToolkit Performance Tools}{hpclink}

\Prog{hpclink} - statically link an application with the \Prog{hpcrun}
profiling code

\end{Name}

%%%%%%%%%%%%%%%%%%%%%%%%%%%%%%%%%%%%%%%%%%%%%%%%%%%%%%%%%%%%%%%%%%
\section{Synopsis}

\Prog{hpclink} \oOpt{options} \Arg{compiler} arg ...

%%%%%%%%%%%%%%%%%%%%%%%%%%%%%%%%%%%%%%%%%%%%%%%%%%%%%%%%%%%%%%%%%%
\section{Description}

\Prog{hpclink} statically links an application with the \Prog{hpcrun}
profiling code.  Dynamically linked binaries can be run directly with
the \Prog{hpcrun} command, but this method does not work with
statically linked programs.  Instead, the \Prog{hpcrun} code must be
linked into the application at build time.

This approach does not require source code modifications, and you
continue to compile your object files as before.  In the application's
Makefile, locate the last step in the build, that is, the command that
produces the final, statically linked binary.  Edit that line to put
the \Prog{hpclink} command at the front of the command line.

The argument list passed to \Prog{hpclink} (\texttt{compiler arg ...})
should be the same command line used to build the application
natively, except that you may wish to rename the binary name
(\texttt{-o output-file}).

See \HTMLhref{hpctoolkit.html}{\Cmd{hpctoolkit}{1}} for an overview of
\textbf{HPCToolkit}.

%%%%%%%%%%%%%%%%%%%%%%%%%%%%%%%%%%%%%%%%%%%%%%%%%%%%%%%%%%%%%%%%%%
\section{Options}

\begin{Description}
\item[\Opt{-h}, \Opt{--help}]
Print help.
%
\item[\Opt{-u}, \Opt{--undefined} symbol]
Pass ``symbol'' to the linker as an undefined symbol.  This option is
rarely needed, but if \Prog{hpclink} fails with an undefined reference
to \texttt{\_\_real\_foo}, then the option ``\texttt{-u foo}'' may
induce the linker to correctly link this symbol.  May be used multiple
times.
%
\item[\Opt{-v}, \Opt{--verbose}]
Verbose output.
\end{Description}


%%%%%%%%%%%%%%%%%%%%%%%%%%%%%%%%%%%%%%%%%%%%%%%%%%%%%%%%%%%%%%%%%%
\section{Examples}

Compile the ``hello, world'' program with \Prog{gcc} and link in the
\Prog{hpcrun} code statically.

\begin{verbatim}
    hpclink gcc -o hello -g -O -static hello.c
\end{verbatim}
%
Link an \Prog{hpcrun}-enabled application from object files and the
math library.

\begin{verbatim}
    hpclink gcc -o myprog -static main.o foo.o ... -lm
\end{verbatim}
%
Make both native and \Prog{hpcrun}-enabled versions of an application
from object files and system libraries with the \Prog{mpixlc}
compiler.  Note that the argument list to the \Prog{hpclink} command
is exactly the command to build the native version except for the name
of the output file.

\begin{verbatim}
    mpixlc -o myprog main.o foo.o ... -lm -lpthread
    hpclink mpixlc -o myprog.hpc main.o foo.o ... -lm -lpthread
\end{verbatim}


%%%%%%%%%%%%%%%%%%%%%%%%%%%%%%%%%%%%%%%%%%%%%%%%%%%%%%%%%%%%%%%%%%
\section{Launching Static Programs}

For dynamically linked binaries, the \Prog{hpcrun} script is used to
launch programs and set environment variables, but on systems with
separate compute nodes, this is often not available.  In this case, the
\texttt{HPCRUN\_EVENT\_LIST} environment variable is used to pass the
profiling events to the \Prog{hpcrun} code.

\begin{verbatim}
    export HPCRUN_EVENT_LIST="PAPI_TOT_CYC@4000000"
    myprog arg ...
\end{verbatim}
%
For example, on a Cray XT system, you might launch a job with a PBS
script such as the following.

\begin{verbatim}
    #!/bin/sh
    #PBS -l size=64
    #PBS -l walltime=01:00:00
    cd $PBS_O_WORKDIR
    export HPCRUN_EVENT_LIST="PAPI_TOT_CYC@4000000 PAPI_L2_TCM@400000"
    aprun -n 64 ./myprog arg ...
\end{verbatim}
% $ Artificially end math mode.
%
The IBM Blue Gene system uses the \texttt{--env} option to pass
environment variables, so you might launch a job with a command
such as the following.

\begin{verbatim}
    qsub -t 60 -n 64 --env HPCRUN_EVENT_LIST="WALLCLOCK@1000" \
        /path/to/myprog arg ...
\end{verbatim}


%%%%%%%%%%%%%%%%%%%%%%%%%%%%%%%%%%%%%%%%%%%%%%%%%%%%%%%%%%%%%%%%%%
\section{Notes}

The command line passed to \Prog{hpclink} must produce a statically
linked binary and the \Prog{hpclink} script will fail if it does not.

With some compilers, eg, IBM's XL compilers and the Pathscale
compilers, interprocedural optimization interferes with
\Prog{hpclink}'s ability to link, causing \Prog{hpclink} to fail.  In
this case, it is necessary to disable interprocedural optimization.
It is not necessary to disable all optimization, just interprocedural
analysis.


%%%%%%%%%%%%%%%%%%%%%%%%%%%%%%%%%%%%%%%%%%%%%%%%%%%%%%%%%%%%%%%%%%
\section{See Also}

\HTMLhref{hpcrun.html}{\Cmd{hpcrun}{1}},
\HTMLhref{hpctoolkit.html}{\Cmd{hpctoolkit}{1}}.

%%%%%%%%%%%%%%%%%%%%%%%%%%%%%%%%%%%%%%%%%%%%%%%%%%%%%%%%%%%%%%%%%%
\section{Version}

Version: \Version\ of \Date.

%%%%%%%%%%%%%%%%%%%%%%%%%%%%%%%%%%%%%%%%%%%%%%%%%%%%%%%%%%%%%%%%%%
\section{License and Copyright}

\begin{description}
\item[Copyright] \copyright\ 2008-2009, Rice University.
\item[License] See \File{README.License}.
\end{description}

%%%%%%%%%%%%%%%%%%%%%%%%%%%%%%%%%%%%%%%%%%%%%%%%%%%%%%%%%%%%%%%%%%
\section{Authors}

\noindent
Rice HPCToolkit Research Group \\
Email: \Email{hpc =at= cs.rice.edu} \\
WWW: \URL{http://hpctoolkit.org}/

\LatexManEnd

\end{document}

%% Local Variables:
%% eval: (add-hook 'write-file-hooks 'time-stamp)
%% time-stamp-start: "setDate{ "
%% time-stamp-format: "%:y/%02m/%02d"
%% time-stamp-end: "}\n"
%% time-stamp-line-limit: 50
%% End:

