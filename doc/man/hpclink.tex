%% $Id$

%%%%%%%%%%%%%%%%%%%%%%%%%%%%%%%%%%%%%%%%%%%%%%%%%%%%%%%%%%%%%%%%%%%%%%%%%%%%%
%%%%%%%%%%%%%%%%%%%%%%%%%%%%%%%%%%%%%%%%%%%%%%%%%%%%%%%%%%%%%%%%%%%%%%%%%%%%%

\documentclass[english]{article}
\usepackage[latin1]{inputenc}
\usepackage{babel}
\usepackage{verbatim}

%% do we have the `hyperref package?
\IfFileExists{hyperref.sty}{
   \usepackage[bookmarksopen,bookmarksnumbered]{hyperref}
}{}

%% do we have the `fancyhdr' or `fancyheadings' package?
\IfFileExists{fancyhdr.sty}{
\usepackage[fancyhdr]{latex2man}
}{
\IfFileExists{fancyheadings.sty}{
\usepackage[fancy]{latex2man}
}{
\usepackage[nofancy]{latex2man}
\message{no fancyhdr or fancyheadings package present, discard it}
}}

%% do we have the `rcsinfo' package?
\IfFileExists{rcsinfo.sty}{
\usepackage[nofancy]{rcsinfo}
\rcsInfo $Id$
\setDate{\rcsInfoLongDate}
}{
\setDate{ 2011/09/19}
\message{package rcsinfo not present, discard it}
}

\setVersionWord{Version:}  %%% that's the default, no need to set it.
\setVersion{=PACKAGE_VERSION=}

%%%%%%%%%%%%%%%%%%%%%%%%%%%%%%%%%%%%%%%%%%%%%%%%%%%%%%%%%%%%%%%%%%%%%%%%%%%%%
%%%%%%%%%%%%%%%%%%%%%%%%%%%%%%%%%%%%%%%%%%%%%%%%%%%%%%%%%%%%%%%%%%%%%%%%%%%%%

\begin{document}

\begin{Name}{1}{hpclink}{The HPCToolkit Performance Tools}{The HPCToolkit Performance Tools}{hpclink}

\Prog{hpclink} is used to link \textbf{HPCToolkit}'s performance measurement library (\HTMLhref{hpcrun.html}{\Cmd{hpcrun}{1}}) into a  statically linked application.

\end{Name}

%%%%%%%%%%%%%%%%%%%%%%%%%%%%%%%%%%%%%%%%%%%%%%%%%%%%%%%%%%%%%%%%%%
\section{Synopsis}

\Prog{hpclink} \oOpt{options} \Arg{link-command}

%%%%%%%%%%%%%%%%%%%%%%%%%%%%%%%%%%%%%%%%%%%%%%%%%%%%%%%%%%%%%%%%%%
\section{Description}

\Prog{hpclink} links HPCToolkit's performance measurement library into a statically linked application.
(\HTMLhref{hpcrun.html}{\Cmd{hpcrun}{1}}'s method for injecting its library into a dynamically linked application will not work with a statically linked applications.)
Although \Prog{hpclink} does require a special link step, it requires neither source code modifications nor changes to compiling individual object files and libraries.

To link with \Prog{hpclink}, first locate the last step in your application's build, that is, the command that produces the final, statically linked binary.
Then, use this command as \Arg{link-command}.
One simple way of accomplishing this is to simply prepend the \Prog{hpclink} command to the front of a Makefile's link line.
You may also wish to rename the binary name (\texttt{-o app-hpclink}).

To control HPCToolkit's performance measurement library during an application's execution, use the following environment variables:

\begin{itemize}
\item \verb+HPCRUN_EVENT_LIST=<event1>[@<period1>];...;<eventN>[@<periodN>]+\\
  Sample using \Arg{event1} through \Arg{eventN} using the respective periods.
  Corresponds to \HTMLhref{hpcrun.html}{\Cmd{hpcrun}{1}}'s -e/--event option.

\item \verb+HPCRUN_TRACE=1+\\
  Enable tracing.
  Corresponds to \HTMLhref{hpcrun.html}{\Cmd{hpcrun}{1}}'s -t/--trace option.

\item \verb+HPCRUN_PROCESS_FRACTION=<frac>+\\
  Measure only a fraction \Arg{frac} of the execution's processses.
  For each process, enable measurement with probability \Arg{frac}, where \Arg{frac} is a real number (0.10) or a fraction (1/10) between 0 and 1.
  Corresponds to \HTMLhref{hpcrun.html}{\Cmd{hpcrun}{1}}'s -f/-fp/--process-fraction option.

\item \verb+HPCRUN_OUT_PATH=<outpath>+\\
  Corresponds to \HTMLhref{hpcrun.html}{\Cmd{hpcrun}{1}}'s -o/--output option.

\end{itemize}



%%%%%%%%%%%%%%%%%%%%%%%%%%%%%%%%%%%%%%%%%%%%%%%%%%%%%%%%%%%%%%%%%%

\section{Arguments}

\begin{Description}
\item[\Arg{link-command}] The link command for producing a statically linked application binary.
It typically has the following form:\\
\SP\SP\SP \Arg{compiler} \Arg{[link-options]} \Arg{object-files} \Arg{libraries}
\end{Description}


\subsection{Options: Informational}

\begin{Description}
\item[\Opt{-v}, \Opt{--verbose}]
Verbose. Displays the original and modified command lines.

\item[\Opt{-V}, \Opt{--version}]
Print version information.

\item[\Opt{-h}, \Opt{--help}]
Print help.
\end{Description}


\subsection{Options: Linking}

\begin{Description}

\item[\Opt{--memleak}]
Include HPCToolkit's memory leak detection libraries.

\item[\OptArg{-u}{symbol}, \OptArg{--undefined}{symbol}]
Pass \Arg{symbol} to the linker as an undefined symbol.
This option is rarely needed, but if \Prog{hpclink} fails with an undefined reference to \texttt{\_\_real\_foo}, then the option ``\texttt{-u foo}'' may induce the linker to correctly link this symbol.
May be used multiple times.

\end{Description}


%%%%%%%%%%%%%%%%%%%%%%%%%%%%%%%%%%%%%%%%%%%%%%%%%%%%%%%%%%%%%%%%%%
\section{Examples}

Compile the ``hello, world'' program with \Prog{gcc} and link in the
\Prog{hpcrun} code statically.

\begin{verbatim}
    hpclink gcc -o hello -g -O -static hello.c
\end{verbatim}
%
Link an \Prog{hpcrun}-enabled application from object files and the
math library.

\begin{verbatim}
    hpclink gcc -o myprog -static main.o foo.o ... -lm
\end{verbatim}
%
Make both native and \Prog{hpcrun}-enabled versions of an application from object files and system libraries with the \Prog{mpixlc} compiler.
Note that the argument list to the \Prog{hpclink} command is exactly the command to build the native version except for the name of the output file.

\begin{verbatim}
    mpixlc -o myprog main.o foo.o ... -lm -lpthread
    hpclink mpixlc -o myprog.hpc main.o foo.o ... -lm -lpthread
\end{verbatim}


%%%%%%%%%%%%%%%%%%%%%%%%%%%%%%%%%%%%%%%%%%%%%%%%%%%%%%%%%%%%%%%%%%
\section{Launching Static Programs}

For dynamically linked binaries, the \Prog{hpcrun} script is used to launch programs and set environment variables, but on systems with separate compute nodes, this is often not available.
In this case, the \texttt{HPCRUN\_EVENT\_LIST} environment variable is used to pass the profiling events to the \Prog{hpcrun} code.

\begin{verbatim}
    export HPCRUN_EVENT_LIST="PAPI_TOT_CYC@4000000"
    myprog arg ...
\end{verbatim}
%
For example, on a Cray XT system, you might launch a job with a PBS
script such as the following.

\begin{verbatim}
    #!/bin/sh
    #PBS -l size=64
    #PBS -l walltime=01:00:00
    cd $PBS_O_WORKDIR
    export HPCRUN_EVENT_LIST="PAPI_TOT_CYC@4000000 PAPI_L2_TCM@400000"
    aprun -n 64 ./myprog arg ...
\end{verbatim}
% $ Artificially end math mode.
%
The IBM Blue Gene system uses the \texttt{--env} option to pass
environment variables, so you might launch a job with a command
such as the following.

\begin{verbatim}
    qsub -t 60 -n 64 --env HPCRUN_EVENT_LIST="WALLCLOCK@1000" \
        /path/to/myprog arg ...
\end{verbatim}


%%%%%%%%%%%%%%%%%%%%%%%%%%%%%%%%%%%%%%%%%%%%%%%%%%%%%%%%%%%%%%%%%%
\section{Notes}

The command line passed to \Prog{hpclink} must produce a statically linked binary and the \Prog{hpclink} script will fail if it does not.

With some compilers, eg, IBM's XL compilers and the Pathscale compilers, interprocedural optimization interferes with \Prog{hpclink}'s ability to link, causing \Prog{hpclink} to fail.
In this case, it is necessary to disable interprocedural optimization.
It is not necessary to disable all optimization, just interprocedural analysis.


%%%%%%%%%%%%%%%%%%%%%%%%%%%%%%%%%%%%%%%%%%%%%%%%%%%%%%%%%%%%%%%%%%
\section{See Also}

\HTMLhref{hpctoolkit.html}{\Cmd{hpctoolkit}{1}}.\\
\HTMLhref{hpcrun.html}{\Cmd{hpcrun}{1}}.

%%%%%%%%%%%%%%%%%%%%%%%%%%%%%%%%%%%%%%%%%%%%%%%%%%%%%%%%%%%%%%%%%%
\section{Version}

Version: \Version\ of \Date.

%%%%%%%%%%%%%%%%%%%%%%%%%%%%%%%%%%%%%%%%%%%%%%%%%%%%%%%%%%%%%%%%%%
\section{License and Copyright}

\begin{description}
\item[Copyright] \copyright\ 2002-2018, Rice University.
\item[License] See \File{README.License}.
\end{description}

%%%%%%%%%%%%%%%%%%%%%%%%%%%%%%%%%%%%%%%%%%%%%%%%%%%%%%%%%%%%%%%%%%
\section{Authors}

\noindent
Mark Krentel \\
Rice HPCToolkit Research Group \\
Email: \Email{hpctoolkit-forum =at= rice.edu} \\
WWW: \URL{http://hpctoolkit.org}.

\LatexManEnd

\end{document}

%% Local Variables:
%% eval: (add-hook 'write-file-hooks 'time-stamp)
%% time-stamp-start: "setDate{ "
%% time-stamp-format: "%:y/%02m/%02d"
%% time-stamp-end: "}\n"
%% time-stamp-line-limit: 50
%% End:
