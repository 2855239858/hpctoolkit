%% $Id$

%%%%%%%%%%%%%%%%%%%%%%%%%%%%%%%%%%%%%%%%%%%%%%%%%%%%%%%%%%%%%%%%%%%%%%%%%%%%%
%%%%%%%%%%%%%%%%%%%%%%%%%%%%%%%%%%%%%%%%%%%%%%%%%%%%%%%%%%%%%%%%%%%%%%%%%%%%%

\documentclass[english]{article}
\usepackage[latin1]{inputenc}
\usepackage{babel}
\usepackage{verbatim}

%% do we have the `hyperref package?
\IfFileExists{hyperref.sty}{
   \usepackage[bookmarksopen,bookmarksnumbered]{hyperref}
}{}

%% do we have the `fancyhdr' or `fancyheadings' package?
\IfFileExists{fancyhdr.sty}{
\usepackage[fancyhdr]{latex2man}
}{
\IfFileExists{fancyheadings.sty}{
\usepackage[fancy]{latex2man}
}{
\usepackage[nofancy]{latex2man}
\message{no fancyhdr or fancyheadings package present, discard it}
}}

%% do we have the `rcsinfo' package?
\IfFileExists{rcsinfo.sty}{
\usepackage[nofancy]{rcsinfo}
\rcsInfo $Id$
\setDate{\rcsInfoLongDate}
}{
\setDate{2018/06/28}
\message{package rcsinfo not present, discard it}
}

\setVersionWord{Version:}  %%% that's the default, no need to set it.
\setVersion{=PACKAGE_VERSION=}

%%%%%%%%%%%%%%%%%%%%%%%%%%%%%%%%%%%%%%%%%%%%%%%%%%%%%%%%%%%%%%%%%%%%%%%%%%%%%
%%%%%%%%%%%%%%%%%%%%%%%%%%%%%%%%%%%%%%%%%%%%%%%%%%%%%%%%%%%%%%%%%%%%%%%%%%%%%

\begin{document}

\begin{Name}{1}{hpcrun}{The HPCToolkit Performance Tools}{The HPCToolkit Performance Tools}{hpcrun:\\ Statistical Profiling}

\Prog{hpcrun} is a call path profiler based on statistical sampling.
It supports multiple sample sources during one execution.
\Prog{hpcrun} profiles complex applications (forks, execs, threads and dynamic linking) and may be used in conjunction with parallel process launchers such as MPICH's \texttt{mpiexec} and SLURM's \texttt{srun}.

See \HTMLhref{hpctoolkit.html}{\Cmd{hpctoolkit}{1}} for an overview of \textbf{HPCToolkit}.


\end{Name}

%%%%%%%%%%%%%%%%%%%%%%%%%%%%%%%%%%%%%%%%%%%%%%%%%%%%%%%%%%%%%%%%%%
\section{Synopsis}

\Prog{hpcrun} \oOpt{profiling-options} \Arg{command} \oOpt{command-arguments}

\Prog{hpcrun} \oOpt{info-options}

%%%%%%%%%%%%%%%%%%%%%%%%%%%%%%%%%%%%%%%%%%%%%%%%%%%%%%%%%%%%%%%%%%
\section{Description}

\Prog{hpcrun} profiles the execution of an arbitrary command \Arg{command} using statistical sampling (rather than instrumentation).
It collects per-thread call path profiles that represent the full calling context of sample points.
Sample points may be generated from multiple simultaneous sampling sources.
\Prog{hpcrun} profiles complex applications that use forks, execs, threads, and dynamic linking/unlinking; it may be used in conjuction with parallel process launchers such as MPICH's \texttt{mpiexec} and SLURM's \texttt{srun}.

To profile a statically linked executable, make sure to link with \HTMLhref{hpclink.html}{\Cmd{hpclink}{1}}.

To configure \Prog{hpcrun}'s sampling sources, specify events and periods using the \texttt{-e/--event} option.
For an event \emph{e} and period \emph{p}, after every \emph{p} instances of \emph{e}, a sample is generated that causes \Prog{hpcrun} to inspect the and record information about the monitored \Arg{command}.

When \Arg{command} terminates, a profile measurement databse will be written to the directory:\\
\\
\SP\SP\SP \Prog{hpctoolkit-}\Arg{command}\Prog{-measurements}[\Prog{-}\Arg{pid}]\\
\\
where \Arg{pid} is an operating system process id, if available.

\Prog{hpcrun} enables you to abort a process and write the partial profiling data to disk by sending the Interrupt signal (INT or Ctrl-C).
This can be extremely useful on long-running or misbehaving applications.


%%%%%%%%%%%%%%%%%%%%%%%%%%%%%%%%%%%%%%%%%%%%%%%%%%%%%%%%%%%%%%%%%%
\section{Arguments}

%% SKW 7/6/18:
%% 
%% The following arguments are implemented in hpcstruct's "Args.cpp" but are deprecated
%% and so not documented here, per JohnMC
%% 
%%     agent-c++
%%     agent-cilk
%%     loop-intvl
%%     loop-fwd-subst
%%     -N / -normalize

Default values for optional arguments are shown in \{\}.

\subsection{Command}

\begin{Description}
\item[\Arg{command}] The command to profile.

\item[\Arg{command-arguments}] Arguments to the command to profile.
\end{Description}

\subsection{Options: Informational}

\begin{Description}
\item[\Opt{-l}, \Opt{-L}, \Opt{--list-events}]
List available events. (N.B.: some may not be profilable)

\item[\Opt{-V}, \Opt{--version}]
Print version information.

\item[\Opt{-h}, \Opt{--help}]
Print help.
\end{Description}

\subsection{Options: Profiling}

\begin{Description}

\item[\Opt{-ds}, \Opt{--delay-sampling}]
Don't start sampling until the application chooses to turn it on.
Use this option to measure only a subset of the application's execution by bracketing interesting regions
with calls to \Prog{hpctoolkit_sampling_start()} and \Prog{hpctoolkit_sampling_stop()}.
Sampling may be turned on and off any number of times during an execution;
and measurements from all sampled regions are aggregated for attribution and display.

\item[\OptArg{-e}{event\Lbr@period\Rbr}, \OptArg{--event}{event\Lbr@period\Rbr}]
Profile the  event with corresponding sample period.
\Arg{event} may be a PAPI event, a Perf event, a native processor event,
or the special event \Prog{WALLCLOCK}.
This option may be given multiple times to profile several events at once;
there may be system-dependent limits on how many events can be profiled simultaneously
and on which events may be combined for profiling.
If no events are given, the default is to profile \Prog{WALLCLOCK@5000}.
For perf event's counter, it is possible to specify
the number of frequency as the sample threshold by
prefixing with f before the number.
For instance, to have 100 samples per second, the period
is: @f100 .\\
N.B.:
\begin{itemize}
  \item The special event \Prog{WALLCLOCK} may be used to profile the actual elapsed time in microseconds
  \item WALLCLOCK and hardware events cannot be mixed.
  \item See the ``Sample sources'' under \textbf{NOTES} for additional details.
\end{itemize}

\item[\OptArg{-c}{number}, \OptArg{--count}{number}]
                       Only available for perf event's counter. This option
                       specifies the event period to sample. It uses the same
                       format of period as the option -e mentioned above.


\item[\OptArg{-p}{level}, \OptArg{--precise-ip}{level}]
                       Specify the precise ip level (used only with perf events):
\begin{itemize}
 \item 0: sample ip can have arbitrary skid
 \item 1: sample ip must have constant skid
 \item 2: sample ip requested to have 0 skid
 \item 3: sample ip must have 0 skid
\end{itemize}
                       NOTE: Some architectures support a precise IP with 0 skid.
                             Incorrect level will unable hpcrun to sample the events.



\item[\OptArg{-f}{frac}, \OptArg{-fp}{frac}, \OptArg{--process-fraction}{frac}]
Measure only a fraction \Arg{frac} of the execution's processes.
For each process, enable measurement of each thread with probability \Arg{frac}, a real number or a fraction (1/10) between 0 and 1.
To minimize perturbations, when measurement for a process is disabled
all threads in a process still receive sampling interrupts but they are ignored.

\item[\OptArg{-lm}{size}, \OptArg{--low-memsize}{size}]
Allocate an additional segment to store measurement data
whenever free space in the current segment is less than the specified \Arg{size}.
If not given, the default for \Arg{size} is~80K.

\item[\OptArg{-m}{size}, \OptArg{--memsize}{size}]
Uuse the specified \Arg{size} as segment size when allocating memory for measurement data.
The specified value is rounded up to a multiple of the `system page size.
If not given, the default for \Arg{size} is~4M.

\item[\OptArg{-mp}{prob}, \OptArg{--memleak-prob}{prob}]
Monitor a subset of memory allocations performed by the application to detect leaks.
An allocation is a call to one of \Prog{malloc}, \Prog{calloc}, \Prog{realloc}, etc\.
and its matching call to \Prog{free}.
At each allocation HPCToolkit generates a pseudo-random number in the range [0.0, 1.0)
and monitors the allocation if the number is less than the value \Arg{prob} specified here,
The value may be written as a a floating point number or as a fraction.
If not given, the default for \Arg{prob} is~0.1.

\item[\OptArg{-o}{outpath}, \OptArg{--output}{outpath}]
Directory to receive output data.
If not given, the default directory ia \Prog{hpctoolkit-<command>-measurements[-<pid>]}.
\begin{itemize}
 Bug: If no \Arg{pid} is available and no output option is given,
 profiles from multiple runs of the same <command>  will be placed into the same output directory.
\end{itemize}

 \item[\Opt{-r}, \Opt{--retain-recursion}]
Do not collapse simple recursive call chains to a single node.
Normally \Prog{hpcrun} does collapse such chains to present a more useful attribution of costs.
If this option is given, all elements of a recursive call chain are recorded.
Note: When you use the \Prog{RETCNT} sample source then this option is enabled automatically
in order to gather accurate counts.

\item[\Opt{-t}, \Opt{--trace}]
Generate a call path trace (in addition to a call path profile).

\end{Description}

\subsection{Options: HPCToolkit Development}

\begin{Description}

These options are intended for use by the HPCToolkit team,
but could be helpful to others interested in HPCToolkit's implementation.
.
 
\item[\Opt{-d}, \Opt{--debug}]
After initialization, spin wait until you attach a debugger
to one or more of the application's processes.
After attaching you can set breakpoints or watchpoints in your application's code
or in HPCToolkit’s \Prog{hpcrun} code before beginning application execution.
To continue after attaching, use the debugger to set program variable \Prog{DEBUGGER WAIT} to zero
and then resume execution.
Note: Your  can only set \Prog{HPCRUN WAIT} if your HPCToolkit was built with debugging symbols.
To build HPCToolkit with debugging symbols,
include the option \Prog{–enable-develop} when configuring HPCToolkit.

\item[\OptArg{-dd}{flag}, \OptArg{--dynamic-debug}{flag}]
Enable the flag \Prog{flag},
causing \Prog{hpcrun} to log debug messages guarded with that flag
during execution.
A list of dynamic debug flags can be found in HPCToolkit’s source code
in the file \Prog{src/tool/hpcrun/messages/messages.flag-defns}.
Note that not all flags are meaningful on all architectures.
The special value \Prog{ALL} enables all debug flags.
\\
Caution: turning on debug flags produces many log messages,
often dramatically slowing the application and potentially distorting the measured profile.

\item[\Opt{-q}, \Opt{--quiet}]
Turn on a default set of dynamic debugging vari\-ables to log information
about HPCToolkit’s stack unwinding based on on-the-fly binary analysis.
See the HPCToolkit User Manual for more details. \\
Bug: this option is unfortunately named.

\item[\Opt{-md}, \Opt{--monitor-debug}]
Enable debug tracing of \Prog{libmonitor}, the \Prog{hpcrun} subsystem which implements process/thread control.
See the HPCToolkit User Manual for more details. 

\end{Description}


%%%%%%%%%%%%%%%%%%%%%%%%%%%%%%%%%%%%%%%%%%%%%%%%%%%%%%%%%%%%%%%%%%

\section{Environment Variables}
For most systems, \Prog{hpcrun} requires no special environment variable settings.
There are two situations, however, where \Prog{hpcrun}, to function correctly,
\emph{must} refer to environment variables. These environment variables, and
corresponding situations are:
\begin{Description}

  \item[\verb+HPCTOOLKIT+] To function correctly, \Prog{hpcrun} must know
       the location of the \Prog{HPCToolkit} top-level installation directory.
       The \Prog{hpcrun} script uses elements of the installation \File{lib} and
       \File{libexec} subdirectories. For most systems, the 
       installation procedure ensures that \Prog{hpcrun} can find the requisite
       components. Some parallel job launchers, however, will \emph{copy} the
       \Prog{hpcrun} script to a different location from the installed base. If your
       system uses this copying mechanism, you must set the \verb+HPCTOOLKIT+
       environment variable to the top-level installation directory.
       
  \item[\verb+hpcrun+] If you refer to the \Prog{hpcrun} script via a file system link
       you must also set \verb+HPCTOOLKIT+, for the same reason.
       
\end{Description}

%%%%%%%%%%%%%%%%%%%%%%%%%%%%%%%%%%%%%%%%%%%%%%%%%%%%%%%%%%%%%%%%%%

\section{Launching}

When sampling with native events, by default hpcrun will profile using perf events.
To force HPCToolkit to use PAPI (assuming it's available) instead of perf events, one
must prefix the event with ‘\texttt{papi::}’ as follows:

\begin{verbatim}
hpcrun -e papi::CYCLES
\end{verbatim}

For PAPI presets, there is no need to prefix the event with ‘papi::’. For instance it is
sufficient to specify \texttt{PAPI\_TOT\_CYC} event without any prefix to profile using PAPI.

To sample an execution 100 times per second (frequency-based sampling) counting
CYCLES and 100 times a second counting INSTRUCTIONS:
\begin{verbatim}
hpcrun -e CYCLES@f100 -e INSTRUCTIONS@f100 ...
\end{verbatim}

To sample an execution every 1,000,000 cycles and every 1,000,000 instructions using
period-based sampling:
\begin{verbatim}
hpcrun -e CYCLES@1000000 -e INSTRUCTIONS@1000000 ...
\end{verbatim}
By default, hpcrun will use frequency-based sampling with the rate 300 samples per
second per event type. Hence the following command will cause HPCToolkit to sample
CYCLES at 300 samples per second and INSTRUCTIONS at 300 samples per second:
\begin{verbatim}
hpcrun -e CYCLES -e INSTRUCTIONS ...
\end{verbatim}
One can a different default rate using the -c option. The command below will sample
CYCLES at 200 samples per second and INSTRUCTIONS at 200 samples per second:
\begin{verbatim}
hpcrun -c f200 -e CYCLES -e INSTRUCTIONS ...
\end{verbatim}

\section{Examples}

Assume we wish to profile the application \texttt{zoo}.
The following examples lists some useful events for different processor architectures.
In each case, the special option \texttt{--} is used to clearly demarcate the end of \Prog{hpcrun} options.

\begin{itemize}
\item \Prog{hpcrun -e CYCLES -e INSTRUCTIONS zoo}
\item \Prog{hpcrun -e WALLCLOCK@5000 zoo}
\item \Prog{hpcrun -e DC_L2_REFILL@1300013 -e PAPI_L2_DCM@510011 -e PAPI_STL_ICY@5300013 -e PAPI_TOT_CYC@13000019 zoo}
\item \Prog{hpcrun -e PAPI_L2_DCM@510011 -e PAPI_TLB_DM@510013 -e PAPI_STL_ICY@5300013 -e PAPI_TOT_CYC@13000019 zoo}
\end{itemize}

\end{enumerate}

%%%%%%%%%%%%%%%%%%%%%%%%%%%%%%%%%%%%%%%%%%%%%%%%%%%%%%%%%%%%%%%%%%
\section{Notes}

\subsection{Sample sources}

%\newcommand{\	exttt{perf\_events}}{{\tt perf\_events}}

\Prog{hpcrun} uses Linux perf\_events (default on Linux platform) and optionally the PAPI library to provide access to hardware performance counter events.
It is important to note that on most out-of-order pipelined architectures, a hardware counter interrupt is not precisely attributed to the instruction that induced the counter to overflow.
The gap is commonly 50-70 instructions.
This means that one should not assume that aggregation at the source line level is fully precise.
(E.g., if a L1 D-cache miss is attributed to a statement that has been compiled to register-only operations, assume the miss is attributed to a nearby load.)
However, aggregation at the procedure and loop level is reliable.


\subsubsection{Linux perf\_events Interface}

Linux \texttt{perf\_events} provides a powerful interface that supports 
measurement of both application execution and kernel activity. 
Using
\texttt{perf\_events}, one can measure both hardware and software events. 
Using a processor's hardware performance monitoring unit (PMU), the
\texttt{perf\_events} interface can measure an execution using any hardware counter
supported by the PMU. Examples of hardware events include cycles, instructions
completed, cache misses, and stall cycles. Using instrumentation built in to the Linux kernel,
the \texttt{perf\_events} interface can measure software events. Examples of software events include page
faults, context switches, and CPU migrations. 


HPCToolkit uses libpfm4 to translate from an event name string to an event code recognized by the kernel. 
An event name is case insensitive and is defined as followed:
\begin{verbatim}
[pmu::][event_name][:unit_mask][:modifier|:modifier=val] 
\end{verbatim}

\begin{itemize}
	\item \textbf{pmu}.  Optional name of the PMU (group of events) to which the event belongs to. This is useful to disambiguate events in case events from difference sources have the same name. If no pmu is specified, the first match event is used.
	\item \textbf{event\_name}.  The name of the event. It must be the complete name, partial matches are not accepted. 
	\item \textbf{unit\_mask}.  This designate an optional sub-events. Some events can be refined using sub-events. An event may have multiple unit masks and it is possible to combine them (for some events) by repeating \texttt{:unit\_mask} pattern.
	\item \textbf{modifier}.  A modifier is an optional filter which modifies how the event counts. Modifiers have a type and a value specified after the equal sign. 
		For boolean type modifiers, without specifying the value, the presence of the modifier is interpreted as meaning true. Events may support multiple modifiers, by repeating the \texttt{:modifier|:modifier=val} pattern. 
	\begin{itemize}
		\item \textbf{precise\_ip}. For some events, it is possible to control the amount of skid. Skid is how many instructions execute between an event of interest happening and the kernel being able to stop and record the event. Smaller skid allows more accurate reporting of which events correspond to which instructions. By default, hpcrun allows arbitrary skid.\\
			The syntax to set the skid modifier:
	\begin{verbatim}
	:p[=k]
	\end{verbatim}
			It is optional to set the value of skid. If no value is specified, HPCToolkit attempts to detect the best skid possible.
			Possible values of k are:
			\begin{itemize}
				\item 0 : sample can have arbitrary skid.
				\item 1 : sample must have constant skid.
				\item 2 : sample requested to have 0 skid.
				\item 3 : sample must have 0 skid.  
			\end{itemize}
			Attention: If the kernel or the hardware does not support the specified value of the skid, no samples will be delivered.\\
	\end{itemize}
\end{itemize}

Some capabilities of HPCToolkit's \texttt{perf\_events} Interface include:
\begin{itemize}
	\item \textbf{Frequency-based sampling.} 
Rather than picking a sample period for a hardware counter,
the Linux \texttt{perf\_events} interface enables one to specify the desired sampling frequency
and have the kernel automatically select and adjust the period 
to try to achieve the desired sampling frequency.
To use frequency-based sampling, one can specify the sampling rate
for an event as the desired number of samples per second
by prefixing the rate with the letter “\verb+f+”. 

\item \textbf{Multiplexing.} 
Using multiplexing enables one to monitor more events
in a single execution than the number of hardware counters a processor
can support for each thread. The number of events that can be monitored in
a single execution is only limited by the maximum number of concurrent
events that the kernel will allow a user to multiplex using the
\texttt{perf\_events} interface.

When more events are specified than can be monitored simultaneously
using a thread's hardware counters,
  the kernel will employ multiplexing and divide
the set of events to be monitored into groups, monitor only one group
of events at a time, and cycle repeatedly through the groups
as a program executes. 


\item \textbf{Thread blocking.} When a program executes, 
a thread may block waiting for the kernel to complete some operation on its behalf.
Example operations include waiting for a \texttt{read} operation to complete or having the
kernel service a page fault or zero-fill a page. On systems running Linux 4.3 or newer, one can use the \texttt{perf\_events} sample source to monitor how much time a thread is blocked and where the blocking occurs. To measure
the time a thread spends blocked, one can profile with \verb+BLOCKTIME+ event and
another time-based event, such as \verb+CYCLES+. The \verb+BLOCKTIME+ event shouldn't have any frequency or period specified, whereas \verb+CYCLES+ should have a frequency or period specified.

\end{itemize}

\subsubsection{PAPI Interface (optional)}
The PAPI library supports a large collection of hardware counter events.
Some events have standard names across all platforms, e.g. \verb+PAPI_TOT_CYC+, the event that measures total cycles.
In addition to events whose names begin with the \verb+PAPI_+ prefix, platforms also provide access to a set of native events with names that are specific to the platform's processor.
A complete list of events supported by the PAPI library for your platform may be obtained by using the \texttt{--list-events} option.
Any event whose name begins with the \verb+PAPI_+ prefix that is listed as "Profilable" can be used as an event in a sampling source --- provided it does not conflict with another event.

The precise rules for selecting good events and periods are complex.
\begin{itemize}

\item \textbf{Choosing sampling events.}
We recommend using PAPI events in general.
However, some PAPI events are not profilable because of PAPI implementation details.
Also, PAPI's standard event list may not cover an architectural feature you are interested in.
In such cases, it is necessary to resort to native events.
In many cases, you will have to consult the architecture's manual to fully understand what the event means: there is no standard event list or naming scheme and events sometimes have unusual meanings.

\item \textbf{Number of sampling events.}
Currently, hpcrun does not multiplex hardware counters.
This means that the number of events that you may concurrently profile against is limited by your architecture's performance monitoring unit.
Note that some architectures hard-wire one or more counters to a specific event (such as cycles).

\item \textbf{Choosing sampling periods.}
The key requirement in choosing sampling periods is that you obtain enough samples to provide statistical significance.
We usually recommend a sampling rate between 100s-1000s of samples per second.
This usually only produces 1-5\% execution time overhead.

Choosing sampling rates depends on the architecture and sometimes the application.

Choosing periods for cycle and instruction-related events are usually easy.
Cycles directly relates to the clock speed.
Instruction-related events relates to the issue rate and width.

Choosing periods for other events seems harder because different applications uses resources differently.
For example, some applications are memory intensive and others are not.
However, if the goal is to identify rate-limiting factors of the architecture, then it is not necessary to consider the application.
For example, if the goal is to determine whether L2 D-cache latency is a limiting factor, then it is only necessary to work backward from the architecture's specifications to determine what number of L2 D-cache misses per second would be problematic.

\item \textbf{Architectural event conflicts.}
With some performance monitoring units, certain events may not be concurrently used with other events.
\end{itemize}

\subsubsection{System itimer (WALLCLOCK).}
On Linux systems, the kernel will not deliver itimer interrupts faster than the unit of a jiffy, which defaults to 4 milliseconds; see the itimer man page.
One can configure the kernel to use a value as small as 1 millisecond, but it is unlikely the kernel will actually deliver itimer signals at that rate when a period of 1000 microseconds is requested.

However, on Linux one can get quite close to the kernel Hz rate by setting the itimer interval to something less than the Hz rate.
For example, if the Hz rate is 1000 microseconds, one can use 500 microseconds (or just 1) and obtain about 999 interrupts per second.

\subsection{Platform-specific notes}
\subsubsection{Cray XE and XK}
When using dynamically linked binaries on Cray XE and XK systems, you
should add the \verb+HPCTOOLKIT+ environment variable to your launch
script.  Set \verb+HPCTOOLKIT+ to the top-level \verb+HPCToolkit+ install
prefix (the directory containing the \File{bin}, \File{lib} and
\File{libexec} subdirectories) and export it to the environment.  This is
only needed for running dynamically linked binaries.  For example:

\begin{verbatim}
#!/bin/sh
#PBS -l mppwidth=#nodes
#PBS -l walltime=00:30:00
#PBS -V

export HPCTOOLKIT=/path/to/hpctoolkit/install/directory

    ...Rest of Script...
\end{verbatim}

If \verb+HPCTOOLKIT+ is not set, you may see errors such as the
following in your job's error log.

\begin{verbatim}
/var/spool/alps/103526/hpcrun: Unable to find HPCTOOLKIT root directory.
Please set HPCTOOLKIT to the install prefix, either in this script,
or in your environment, and try again.
\end{verbatim}

The problem is that the Cray job launcher copies the \Prog{hpcrun}
script to a directory somewhere below \File{/var/spool/alps/} and runs
it from there.  By moving \Prog{hpcrun} to a different directory, this
breaks \Prog{hpcrun}'s method for finding its own install directory.
The solution is to add \verb+HPCTOOLKIT+ to your environment so that
\Prog{hpcrun} can find its install directory.

\subsection{Miscellaneous}

\begin{itemize}
  \item \Prog{hpcrun} uses preloaded shared libraries to initiate profiling.  For this reason, it cannot be used to profile setuid programs.
  \item \Prog{hpcrun} may not be able to profile programs that themselves use preloading.
\end{itemize}


%%%%%%%%%%%%%%%%%%%%%%%%%%%%%%%%%%%%%%%%%%%%%%%%%%%%%%%%%%%%%%%%%%
\section{See Also}

\HTMLhref{hpctoolkit.html}{\Cmd{hpctoolkit}{1}}.\\
\HTMLhref{hpclink.html}{\Cmd{hpclink}{1}}.


%%%%%%%%%%%%%%%%%%%%%%%%%%%%%%%%%%%%%%%%%%%%%%%%%%%%%%%%%%%%%%%%%%
\section{Version}

Version: \Version

%%%%%%%%%%%%%%%%%%%%%%%%%%%%%%%%%%%%%%%%%%%%%%%%%%%%%%%%%%%%%%%%%%
\section{License and Copyright}

\begin{description}
\item[Copyright] \copyright\ 2002-2018, Rice University.
\item[License] See \File{README.License}.
\end{description}

%%%%%%%%%%%%%%%%%%%%%%%%%%%%%%%%%%%%%%%%%%%%%%%%%%%%%%%%%%%%%%%%%%
\section{Authors}

\noindent
Rice University's HPCToolkit Research Group \\
Email: \Email{hpctoolkit-forum =at= rice.edu} \\
WWW: \URL{http://hpctoolkit.org}.

\LatexManEnd

\end{document}

%% Local Variables:
%% eval: (add-hook 'write-file-hooks 'time-stamp)
%% time-stamp-start: "setDate{ "
%% time-stamp-format: "%:y/%02m/%02d"
%% time-stamp-end: "}\n"
%% time-stamp-line-limit: 50
%% End:

