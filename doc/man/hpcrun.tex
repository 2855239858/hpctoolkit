%% $Id$

%%%%%%%%%%%%%%%%%%%%%%%%%%%%%%%%%%%%%%%%%%%%%%%%%%%%%%%%%%%%%%%%%%%%%%%%%%%%%
%%%%%%%%%%%%%%%%%%%%%%%%%%%%%%%%%%%%%%%%%%%%%%%%%%%%%%%%%%%%%%%%%%%%%%%%%%%%%

\documentclass[english]{article}
\usepackage[latin1]{inputenc}
\usepackage{babel}
\usepackage{verbatim}

%% do we have the `hyperref package?
\IfFileExists{hyperref.sty}{
   \usepackage[bookmarksopen,bookmarksnumbered]{hyperref}
}{}

%% do we have the `fancyhdr' or `fancyheadings' package?
\IfFileExists{fancyhdr.sty}{
\usepackage[fancyhdr]{latex2man}
}{
\IfFileExists{fancyheadings.sty}{
\usepackage[fancy]{latex2man}
}{
\usepackage[nofancy]{latex2man}
\message{no fancyhdr or fancyheadings package present, discard it}
}}

%% do we have the `rcsinfo' package?
\IfFileExists{rcsinfo.sty}{
\usepackage[nofancy]{rcsinfo}
\rcsInfo $Id$
\setDate{\rcsInfoLongDate}
}{
\setDate{ 2009/01/13}
\message{package rcsinfo not present, discard it}
}

\setVersionWord{Version:}  %%% that's the default, no need to set it.
\setVersion{=PACKAGE_VERSION=}

%%%%%%%%%%%%%%%%%%%%%%%%%%%%%%%%%%%%%%%%%%%%%%%%%%%%%%%%%%%%%%%%%%%%%%%%%%%%%
%%%%%%%%%%%%%%%%%%%%%%%%%%%%%%%%%%%%%%%%%%%%%%%%%%%%%%%%%%%%%%%%%%%%%%%%%%%%%

\begin{document}

\begin{Name}{1}{hpcrun}{The HPCToolkit Performance Tools}{The HPCToolkit Performance Tools}{hpcrun:\\ Statistical Profiling}

\Prog{hpcrun} is a call path profiler based on statistical sampling.
It supports multiple sample sources during one execution.
\Prog{hpcrun} profiles complex applications (forks, execs, threads and dynamically loaded libraries) and may be used in conjunction with parallel process launchers such as MPICH's \texttt{mpiexec} and SLURM's \texttt{srun}.

See \HTMLhref{hpctoolkit.html}{\Cmd{hpctoolkit}{1}} for an overview of \textbf{HPCToolkit}.


\end{Name}

%%%%%%%%%%%%%%%%%%%%%%%%%%%%%%%%%%%%%%%%%%%%%%%%%%%%%%%%%%%%%%%%%%
\section{Synopsis}

\Prog{hpcrun} \oOpt{profiling-options} \Arg{command} \oOpt{command-arguments}

%%%%%%%%%%%%%%%%%%%%%%%%%%%%%%%%%%%%%%%%%%%%%%%%%%%%%%%%%%%%%%%%%%
\section{Description}

\Prog{hpcrun} profiles the execution of an arbitrary command \Arg{command} using statistical sampling.
It creates call path profiles and supports multiple sample sources during one execution.
Specifically, for an event \emph{e} and period \emph{p}, after every \emph{p} instances of \emph{e} a call path sample is collected and stored.
\Prog{hpcrun} profiles complex applications (forks, execs, threads and dynamically loaded libraries) and may be used in conjunction with parallel process launchers such as MPICH's \texttt{mpiexec} and SLURM's \texttt{srun}.

When \Arg{command} terminates normally, a profile will be written to a file with the name\\
\SP\SP\SP \Arg{command}.\Arg{hostname}.\Arg{pid}.\Arg{tid}.csp


%%%%%%%%%%%%%%%%%%%%%%%%%%%%%%%%%%%%%%%%%%%%%%%%%%%%%%%%%%%%%%%%%%
\section{Arguments}

\begin{Description}
\item[\Arg{command}] The command to profile.
\item[\Arg{command-arguments}] Arguments to the command to profile.
\end{Description}

Default values for an option's optional arguments are shown in \{\}.

\subsection{Options: Informational}

\begin{Description}
\item[\Opt{-V}, \Opt{--version}] Print version information.
\item[\Opt{-h}, \Opt{--help}] Print help.
\end{Description}

\subsection{Options: Profiling}

\begin{Description}
  \item[\OptArg{-e}{event\Lbr:period\Rbr}]  An event to profile and its corresponding sample period. \Arg{event} may be either a PAPI or native processor event. \{\verb+WALLCLOCK:5000+\} May pass multiple times. 
  \begin{itemize}
    \item It is recommended to always specify the sampling period for each profiling event.
    \item The special event WALLCLOCK may be used to profile the `wall clock.'  It may be used only \emph{once} and cannot be used with another event. It is an error to specify a period.
    \item Hardware and drivers 1) limit the maximum number of events that can be monitored simultaneously, and 2) forbid certain combinations of events. Check your documentation.
  \end{itemize}
  \item[\OptoArg{-o}{outpath}, \OptoArg{--output}{outpath}] Directory for output data.  \{\texttt{.}\}
\end{Description}


%%%%%%%%%%%%%%%%%%%%%%%%%%%%%%%%%%%%%%%%%%%%%%%%%%%%%%%%%%%%%%%%%%
\section{Examples}

Assume we wish to profile the application \texttt{zoo}.
The following examples lists some useful events for different processor architectures.
In each case, the special option \texttt{--} is used to clearly demarcate the end of \Prog{hpcrun} options.

\begin{enumerate}
\item \verb+hpcrun -e WALLCLOCK:5000 zoo+

\item Opteron, (Rev B-F)
  \begin{enumerate}
    \item \verb+hpcrun -e DC_L2_REFILL:1300013 -e PAPI_L2_DCM:510011 -e PAPI_STL_ICY:5300013 -e PAPI_TOT_CYC:13000019 zoo+ (\verb+DC_L2_REFILL+ is an approximation of L1 D-cache misses).
    \item \verb+hpcrun -e PAPI_L2_DCM:510011 -e PAPI_TLB_DM:510013 -e PAPI_STL_ICY:5300013 -e PAPI_TOT_CYC:13000019 zoo+
  \end{enumerate}

\end{enumerate}

%%%%%%%%%%%%%%%%%%%%%%%%%%%%%%%%%%%%%%%%%%%%%%%%%%%%%%%%%%%%%%%%%%
\section{Notes}

\begin{itemize}
  \item Because hpcrun uses \verb+LD_PRELOAD+ to initiate profiling, it cannot be used to profile setuid commands.
\end{itemize}


%%%%%%%%%%%%%%%%%%%%%%%%%%%%%%%%%%%%%%%%%%%%%%%%%%%%%%%%%%%%%%%%%%
\section{See Also}

\HTMLhref{hpctoolkit.html}{\Cmd{hpctoolkit}{1}}.

%%%%%%%%%%%%%%%%%%%%%%%%%%%%%%%%%%%%%%%%%%%%%%%%%%%%%%%%%%%%%%%%%%
\section{Version}

Version: \Version\ of \Date.

%%%%%%%%%%%%%%%%%%%%%%%%%%%%%%%%%%%%%%%%%%%%%%%%%%%%%%%%%%%%%%%%%%
\section{License and Copyright}

\begin{description}
\item[Copyright] \copyright\ 2002-2008, Rice University.
\item[License] See \File{README.License}.
\end{description}

%%%%%%%%%%%%%%%%%%%%%%%%%%%%%%%%%%%%%%%%%%%%%%%%%%%%%%%%%%%%%%%%%%
\section{Authors}

\noindent
Nathan Tallent \\
John Mellor-Crummey \\
Rob Fowler \\
Email: \Email{hpc@cs.rice.edu} \\
WWW: \URL{http://hipersoft.cs.rice.edu/hpctoolkit}.

\LatexManEnd

\end{document}

%% Local Variables:
%% eval: (add-hook 'write-file-hooks 'time-stamp)
%% time-stamp-start: "setDate{ "
%% time-stamp-format: "%:y/%02m/%02d"
%% time-stamp-end: "}\n"
%% time-stamp-line-limit: 50
%% End:

