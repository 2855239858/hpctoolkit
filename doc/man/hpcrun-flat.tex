%% $Id$

%%%%%%%%%%%%%%%%%%%%%%%%%%%%%%%%%%%%%%%%%%%%%%%%%%%%%%%%%%%%%%%%%%%%%%%%%%%%%
%%%%%%%%%%%%%%%%%%%%%%%%%%%%%%%%%%%%%%%%%%%%%%%%%%%%%%%%%%%%%%%%%%%%%%%%%%%%%

\documentclass[english]{article}
\usepackage[latin1]{inputenc}
\usepackage{babel}
\usepackage{verbatim}

%% do we have the `hyperref package?
\IfFileExists{hyperref.sty}{
   \usepackage[bookmarksopen,bookmarksnumbered]{hyperref}
}{}

%% do we have the `fancyhdr' or `fancyheadings' package?
\IfFileExists{fancyhdr.sty}{
\usepackage[fancyhdr]{latex2man}
}{
\IfFileExists{fancyheadings.sty}{
\usepackage[fancy]{latex2man}
}{
\usepackage[nofancy]{latex2man}
\message{no fancyhdr or fancyheadings package present, discard it}
}}

%% do we have the `rcsinfo' package?
\IfFileExists{rcsinfo.sty}{
\usepackage[nofancy]{rcsinfo}
\rcsInfo $Id$
\setDate{\rcsInfoLongDate}
}{
\setDate{ 2008/07/09}
\message{package rcsinfo not present, discard it}
}

\setVersionWord{Version:}  %%% that's the default, no need to set it.
\setVersion{=PACKAGE_VERSION=}

%%%%%%%%%%%%%%%%%%%%%%%%%%%%%%%%%%%%%%%%%%%%%%%%%%%%%%%%%%%%%%%%%%%%%%%%%%%%%
%%%%%%%%%%%%%%%%%%%%%%%%%%%%%%%%%%%%%%%%%%%%%%%%%%%%%%%%%%%%%%%%%%%%%%%%%%%%%

\begin{document}

\begin{Name}{1}{hpcrun-flat}{The HPCToolkit Performance Tools}{The HPCToolkit Performance Tools}{hpcrun-flat:\\ Statistical Profiling}

\Prog{hpcrun-flat} is a flat statistical sampling-based profiler.
It supports multiple sample sources during one execution and creates an IP (instruction pointer) histogram, or flat profile, for each sample source.
\Prog{hpcrun-flat} profiles complex applications (forks, execs, threads and dynamically loaded libraries) and may be used in conjunction with parallel process launchers such as MPICH's \texttt{mpiexec} and SLURM's \texttt{srun}.

See \HTMLhref{hpctoolkit.html}{\Cmd{hpctoolkit}{1}} for an overview of \textbf{HPCToolkit}.


\end{Name}

%%%%%%%%%%%%%%%%%%%%%%%%%%%%%%%%%%%%%%%%%%%%%%%%%%%%%%%%%%%%%%%%%%
\section{Synopsis}

\Prog{hpcrun-flat} \oOpt{profiling-options} [\texttt{--}] \Arg{command} \oOpt{command-arguments}

\Prog{hpcrun-flat} \oOpt{info-options}

%%%%%%%%%%%%%%%%%%%%%%%%%%%%%%%%%%%%%%%%%%%%%%%%%%%%%%%%%%%%%%%%%%
\section{Description}

\Prog{hpcrun-flat} profiles the execution of an arbitrary command \Arg{command} using statistical sampling.
It supports multiple sample sources during one execution and creates an IP (instruction pointer) histogram, or flat profile, for each sample source.
Specifically, for an event \emph{e} and period \emph{p}, after every \emph{p} instances of \emph{e} a counter associated with the instruction of the current IP is incremented.
\Prog{hpcrun-flat} profiles complex applications (forks, execs, threads and dynamically loaded libraries) and may be used in conjunction with parallel process launchers such as MPICH's \texttt{mpiexec} and SLURM's \texttt{srun}.

When \Arg{command} terminates normally, a profile --- a histogram of counts for instructions in each load module --- will be written to a file with the name\\
\SP\SP\SP \Arg{command}.hpcrun-flat.\Arg{hostname}.\Arg{pid}.\Arg{tid}\\
\Prog{hpcrun-flat} allows the user to abort a process \emph{and} write the partial profiling data to disk by sending the Interrupt signal (INT or Ctrl-C).  This can be extremely useful on long-running or misbehaving applications.

The special option `\texttt{--}' can be used to stop \Prog{hpcrun-flat} option parsing; this is especially useful when \Arg{command} takes arguments of its own.


%%%%%%%%%%%%%%%%%%%%%%%%%%%%%%%%%%%%%%%%%%%%%%%%%%%%%%%%%%%%%%%%%%
\section{Arguments}

\begin{Description}
\item[\Arg{command}] The command to profile.
\item[\Arg{command-arguments}] Arguments to the command to profile.
\end{Description}

Default values for an option's optional arguments are shown in \{\}.

\subsection{Options: Informational}

\begin{Description}
\item[\Opt{-l}, \Opt{--events-short}] List available events (N.B.: some may not be profilable)
\item[\Opt{-L}, \Opt{--events-long}] Similar to above but with more information.
\item[\Opt{--paths}] Print paths for external PAPI and MONITOR.
\item[\Opt{-V}, \Opt{--version}] Print version information.
\item[\Opt{-h}, \Opt{--help}] Print help.
\item[\OptoArg{--debug}{n}]   Debug: use debug level \Arg{n}. \{1\}
\end{Description}

\subsection{Options: Profiling}

\begin{Description}
  \item[\OptoArg{-r}{yes \Bar no}, \OptoArg{--recursive}{yes \Bar no}] Profile processes spawned by \Arg{command}.  \{no\}.  (Each process will receive its own output file.)
  \item[\OptArg{-t}{mode}, \OptArg{--threads}{mode}] Select thread profiling mode \{each\}:
  \begin{itemize}
    \item \textbf{each} Create separate profiles for each thread.
    \item \textbf{all} Create one combined profile of all threads.
  \end{itemize}
  Note that only POSIX threads are supported.
  Also note that the WALLCLK event cannot be used in a multithreaded process.
  \item[\OptArg{-e}{event\Lbr:period\Rbr}, \OptArg{--event}{event\Lbr:period\Rbr}]  An event to profile and its corresponding sample period. \Arg{event} may be either a PAPI or native processor event. \{\verb+PAPI_TOT_CYC:999999+\} May pass multiple times. 
  \begin{itemize}
    \item It is recommended to always specify the sampling period for each profiling event.
    \item The special event WALLCLK may be used to profile the `wall clock.'  It may be used only \emph{once} and cannot be used with another event. It is an error to specify a period.
    \item Hardware and drivers 1) limit the maximum number of events that can be monitored simultaneously, and 2) forbid certain combinations of events. Check your documentation.
  \end{itemize}
  \item[\OptoArg{-o}{outpath}, \OptoArg{--output}{outpath}] Directory for output data.  \{\texttt{.}\}
  \item[\OptArg{--papi-flag}{flag}] Profile style flag.  \{\verb+PAPI_POSIX_PROFIL+\}
\end{Description}


%%%%%%%%%%%%%%%%%%%%%%%%%%%%%%%%%%%%%%%%%%%%%%%%%%%%%%%%%%%%%%%%%%
\section{Examples}

Assume we wish to profile the application \texttt{zoo}.
The following examples lists some useful events for different processor architectures.
In each case, the special option \texttt{--} is used to clearly demarcate the end of \Prog{hpcrun-flat} options.

\begin{enumerate}
\item \verb+hpcrun-flat -e WALLCLK -- zoo+
\item Opteron, (Rev B-F)
  \begin{enumerate}
    \item \verb+hpcrun-flat -e DC_L2_REFILL:1300013 -e PAPI_L2_DCM:510011 -e PAPI_STL_ICY:5300013 -e PAPI_TOT_CYC:13000019 -- zoo+ (\verb+DC_L2_REFILL+ is an approximation of L1 D-cache misses).
    \item \verb+hpcrun-flat -e PAPI_L2_DCM:510011 -e PAPI_TLB_DM:510013 -e PAPI_STL_ICY:5300013 -e PAPI_TOT_CYC:13000019 -- zoo+
  \end{enumerate}

\item Pentium IV 
  \begin{enumerate}
    \item \verb+hpcrun-flat -e PAPI_TOT_CYC:30000001 -e PAPI_TOT_INS:3000001 -e PAPI_FP_INS:1000001 -e PAPI_LD_INS:1000001 -e PAPI_TLB_TL:32767  -e PAPI_L2_TCM:32767  -e PAPI_RES_STL:1000001 -e BSQ_cache_reference_RD_3rdL_MISS -- zoo+
    \item \verb+hpcrun-flat -e PAPI_SR_INS:1000001 -e PAPI_L1_DCM:32767 -e resource_stall_SBFULL:32767 -- zoo+
    \item \verb+hpcrun-flat -e PAPI_FP_OPS:32767 -e PAPI_BR_MSP:32767 -- zoo+
  \end{enumerate}

\item Itanium 2
  \begin{enumerate}
    \item \verb+hpcrun-flat -e BE_EXE_BUBBLE_ALL:344221 -e BE_L1D_FPU_BUBBLE_ALL:344221 -e FE_BUBBLE_ALL:144221 -e PAPI_TOT_CYC:344221 -- zoo+
    \item \verb+hpcrun-flat -e PAPI_L1_DCM:144221 -e PAPI_FP_OPS:344221 -e PAPI_TOT_CYC:1044221 -- zoo+
    \item Cycle accounting events: \verb+BACK_END_BUBBLE_ALL+, \verb+BE_EXE_BUBBLE_ALL+, \verb+BE_L1D_FPU_BUBBLE_ALL+, \verb+BE_FLUSH_BUBBLE_ALL+, \verb+BE_RSE_BUBBLE_ALL+, \verb+FE_BUBBLE_ALL+
  \end{enumerate}

\end{enumerate}

%%%%%%%%%%%%%%%%%%%%%%%%%%%%%%%%%%%%%%%%%%%%%%%%%%%%%%%%%%%%%%%%%%
\section{Notes}

\begin{itemize}
  \item Because hpcrun-flat uses \verb+LD_PRELOAD+ to initiate profiling, it cannot be used to profile setuid commands.
  \item For the same reason, it cannot profile statically linked applications.
  \item Bug: For non-recursive profiling, \verb+LD_PRELOAD+ is currently unsetenv'd.  Child processes that otherwise depend \verb+LD_PRELOAD+ will likely die.
\end{itemize}


%%%%%%%%%%%%%%%%%%%%%%%%%%%%%%%%%%%%%%%%%%%%%%%%%%%%%%%%%%%%%%%%%%
\section{See Also}

\HTMLhref{hpctoolkit.html}{\Cmd{hpctoolkit}{1}}.

%%%%%%%%%%%%%%%%%%%%%%%%%%%%%%%%%%%%%%%%%%%%%%%%%%%%%%%%%%%%%%%%%%
\section{Version}

Version: \Version\ of \Date.

%%%%%%%%%%%%%%%%%%%%%%%%%%%%%%%%%%%%%%%%%%%%%%%%%%%%%%%%%%%%%%%%%%
\section{License and Copyright}

\begin{description}
\item[Copyright] \copyright\ 2002-2007, Rice University.
\item[License] See \File{README.License}.
\end{description}

%%%%%%%%%%%%%%%%%%%%%%%%%%%%%%%%%%%%%%%%%%%%%%%%%%%%%%%%%%%%%%%%%%
\section{Authors}

\noindent
Nathan Tallent \\
John Mellor-Crummey \\
Rob Fowler \\
Email: \Email{hpc@cs.rice.edu} \\
WWW: \URL{http://hipersoft.cs.rice.edu/hpctoolkit}.

\LatexManEnd

\end{document}

%% Local Variables:
%% eval: (add-hook 'write-file-hooks 'time-stamp)
%% time-stamp-start: "setDate{ "
%% time-stamp-format: "%:y/%02m/%02d"
%% time-stamp-end: "}\n"
%% time-stamp-line-limit: 50
%% End:

